\documentclass[12pt]{article}
\usepackage[english]{babel}
\usepackage[numbers]{natbib}
\usepackage{graphicx}
\usepackage{xcolor}
\usepackage{sectsty}
\usepackage{float}
\bibliographystyle{apalike}
\setcitestyle{open={[},close={]}}
\sectionfont{\color{DarkBlue}} 
\subsectionfont{\color{LightBlue}}
\subsubsectionfont{\color{LightBlue}}
\paragraphfont{\color{LightBlue}}
\subparagraphfont{\color{LightBlue}}
\begin{document}
\definecolor{DarkBlue}{HTML}{4a5a8a} 
\definecolor{LightBlue}{HTML}{4f81bf}
\begin{titlepage}
	\begin{flushleft}
		\vspace{1cm} \Huge  \textbf{Cretaceous Gardens Controller}\\
		\vspace{1cm} \Huge  \textit{Software Requirements Specification}\\
		\vspace{1cm} \Large \textit{SRS Version 1.0}\\
		\vspace{5cm} \LARGE         Team \#3\\ 
		                            29 October 2019
		\vfill       \Huge  \textbf{CS 460 Software Engineering}
	\end{flushleft}
\end{titlepage}
\normalsize \tableofcontents
\pagebreak

\section{Introduction} \label{intro} % Zeke
% project goals and the purpose of this document
	\paragraph{} The purpose of this document is to \textit{specify} the requirements for the
	 development of the Cretaceous Gardens Controller (CGC). The specification is formalized and 
	 diagrammed in order to guide the eventual implementation of the system. Information 
	 encountered in the corresponding \textit{Requirements Definition Document} is reiterated 
	 and restated here where relevant.
	  
	 \paragraph{} After this introduction \footnote{Introduction by Ezequiel Ramos}, 
	 Section \ref{gen} gives an overview of the system. Section \ref{spec} delves into more 
	 detail with subsections \ref{logic} and \ref{inter} that feature a more granular view 
	 of the \textit{Control Logic} and the \textit{External Interfaces}. Section \ref{defs} 
	 provides the definition of technical terms that will be commonly used.
\pagebreak
\section{General Description} \label{gen} %Zeke
	\paragraph{}\textit{This section \footnote{General Description by Ezequiel Ramos and 
	Santiago Cejas} provides a general overview of the whole system. How the system interacts with 
	the hardware interfaces and its basic functionality are introduced here. A description of
	parts to be used in the system and the available functionalities for each type are also 
	provided. Some high level constraints and assumptions for the system will be also be presented.
	It should be noted that a more detailed specification of constraints is covered in its 
	own section.}
	
	\subsection{Product Perspective}
		\paragraph{} 

		\paragraph{} 
		
		\begin{figure}[H]
  			\centerline{\includegraphics[scale=.52]{Block_Diagram.png}}
  			\caption{Block Diagram}
  			\label{fig:block}
		\end{figure}
		
		\paragraph{} 
		\paragraph{} 	
		\paragraph{} 

	\subsection{Product Functions}
		\paragraph{} 
		
		\paragraph{} 
		
		\paragraph{} 
		
		\paragraph{} 

	\subsection{User Features}
		\paragraph{} 
		
		\paragraph{} 
		
		\paragraph{} 
	
	\subsection{High Level Constraints}
		\paragraph{}

	\subsection{Assumptions}
	\paragraph{} We assume that the infrastructure is all redundant. The CGC is 
	installed on redundant servers. The network backbone has physical redundant 
	links to appropriate devices like the cameras, the PA speakers, and the 
	electric fence. We will also	program redundancy into the logic. Like the 
	ability to have another car available in case of an emergency or if the car 
	breaks down.
		
	\paragraph{} Another assumption is that messages would be encrypted in order 
	to provide the security needed, so the messages can not be intercepted and 
	modified.

\section{Specific Requirements} \label{spec} % Siri and Anas
% logical diagram
\paragraph{} \textit{Section Introduction}

	\subsection{Interfaces} \label{inter}% Anas
	\paragraph{} The Interfaces \footnote{External Interfaces by Anas Gauba} 
	make up all the pieces that the CGC communicates with. The CGC itself must 
	communicate with everything, 	but a lot of interfaces can function on their own. The 
	car interface is an example of one that needs to be able to function on it's own.
		
	\paragraph{Pay Kiosk}
	\textit{}
	    \subparagraph{Incoming Events}
		\begin{enumerate}
			\item 
		\end{enumerate}
				
	    \subparagraph{Outgoing Events}
		\begin{enumerate}
			\item 
		\end{enumerate}

	\paragraph{Token}
	\textit{}
	    \subparagraph{Incoming Events}
		\begin{enumerate}
	        	\item 
		\end{enumerate}
				
	    \subparagraph{Outgoing Events}
		\begin{enumerate}
			\item 
		\end{enumerate}

	\paragraph{Car}
	\textit{}
	    \subparagraph{Incoming Events}
		\begin{enumerate}
			\item 
		\end{enumerate}
				
	    \subparagraph{Outgoing Events}
		\begin{enumerate}
			\item 
		\end{enumerate}

    \paragraph{T-Rex Monitor}
	\textit{}
	    \subparagraph{Incoming Events}
		\begin{enumerate}
			\item 
		\end{enumerate}
				
	    \subparagraph{Outgoing Events}
		\begin{enumerate}
			\item 
		\end{enumerate}

	\paragraph{Camera Network}
	\textit{}
	    \subparagraph{Incoming Events}
		\begin{enumerate}
			\item 
		\end{enumerate}
		
	    \subparagraph{Outgoing Events}
		\begin{enumerate}
			\item 
		\end{enumerate}

	\paragraph{Electric Fence}
	\textit{}
	    \subparagraph{Incoming Events}
		\begin{enumerate}
			\item 
		\end{enumerate}
				
	    \subparagraph{Outgoing Events}
		\begin{enumerate}
			\item 
		\end{enumerate}

	\paragraph{Global Alarm System}
	\textit{}
	    \subparagraph{Incoming Events}
		\begin{enumerate}
			\item 
		\end{enumerate}
				
	    \subparagraph{Outgoing Events}
		\begin{enumerate}
			\item 
		\end{enumerate}

	\paragraph{CGC Station}
	\textit{}
	    \subparagraph{Incoming Events}
		\begin{enumerate}
			\item 
		\end{enumerate}
				
	    \subparagraph{Outgoing Events}
		\begin{enumerate}
			\item 
		\end{enumerate}

	\paragraph{GPS Server}
	\textit{}
	    \subparagraph{Incoming Events}
		\begin{enumerate}
			\item 
		\end{enumerate}
				
	    \subparagraph{Outgoing Events}
		\begin{enumerate}
			\item 
		\end{enumerate}
				
		
		
    \subsection{Control Logic} \label{logic}%siri
	\paragraph{} \textit{ \footnote{Control Logic by Siri Khalsa}}

    \begin{figure}[H]
  		\centerline{\includegraphics[scale=1]{ECSModes.jpg}}
  		\caption{Elevator Control System Normal Function Model }
  		\label{fig:normal}
	\end{figure}

	\begin{figure}[H]
  	    \centerline{\includegraphics[scale=1]{ECS-Normal.jpg}}
  		\caption{Elevator Control System Normal Function Model }
  		\label{fig:normal}
	\end{figure}
	
	\begin{figure}[H]
  		\centerline{\includegraphics[scale=1]{EmergencyPhase1.jpg}}
  		\caption{Elevator Control System Normal Function Model }
  		\label{fig:emerg}
	\end{figure}

\section{Design Constraints} \label{cons} %
% error handling, exceptions, hazards, 
\paragraph{} \textit{Section Intro}

	\subsection{Client}
	\begin{itemize}
		\item 
	\end{itemize}

	\subsection{Safety}
	\begin{itemize}
		\item 
	\end{itemize}

	\subsection{Regulations}
	\begin{itemize}
		\item 
	\end{itemize}

	\subsection{Security}
	\begin{itemize}
		\item 
	\end{itemize}

	% From my researching into SRS papers. The constraints section is usually short and sometimes included in other sections. On top of all that it seems to only pertain to software so I attempted to only include constraints that would pertain to the software. Feel free to add or subtract if you can think of more/better ones.

\section{Definition of Terms} \label{defs} %Anas
	\paragraph{} \textit{The following is a list of definitions \footnote{Definition of 
	Terms by Anas Gauba} of the most commonly used technical terms within this 
	document, whose meaning may not be immediately apparent to the lay reader. Most 
	definitions come from no specific source; instead they are defined by the authors 
	in the context of their use in this document and originate from the vocabulary 
	shared across the general references cited \nocite{*}. In the event that a 
	definition was taken directly from a source, it is followed by a citation}
	
	\begin{list}{}{}
	    \item \textbf{CGC:} Acronym for Cretaceous Gardens Controller 
	    \item \textbf{DVR:} Acronym for Digital Video Recorder
	    \item \textbf{Electrical Conduction:} The movement of electrically charged     
	        particles through a transmission medium.
	    \item \textbf{GPS:} Global Positioning System 
	    \item \textbf{Hardwired Ethernet:} This references the latest IEEE standard for 
	        Ethernet utilizing physical cables.
	    \item \textbf{Network:} All nodes with which the CGC interacts, the links that 
	        connect them to each other and to the
	        CGC, the CGC itself, and all related databases.
	    \item \textbf{Node:} The generic term that refers to any device connected to 
	        the CGC in any way. This includes 
	        autonomous vehicles, tokens, the T.Rex monitor, all electric fence panels, 
	        all kiosks, and all cameras.
	    \item \textbf{Safely Inactive:} A state in which a vehicle is fully functional 
	        and ready to be dispatched.
	    \item \textbf{Safely Occupied:} A state in which a vehicle contains at least 
	        one person, is locked, and is ready to depart.
	    \item \textbf{Token:} An interactive device used by the visitor that grants 
	        access to locations.
	\end{list}
\pagebreak
\bibliography{../../ReferenceMaterial/BibTeX/references} % Zeke
% run latex, then bibtex, then quickbuild (on the tex file)
\end{document}
