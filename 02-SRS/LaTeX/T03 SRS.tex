\documentclass[12pt]{article}
\usepackage[english]{babel}
\usepackage[numbers]{natbib}
\usepackage{graphicx}
\usepackage{xcolor}
\usepackage{sectsty}
\usepackage{float}
\bibliographystyle{apalike}
\setcitestyle{open={[},close={]}}
\sectionfont{\color{DarkBlue}} 
\subsectionfont{\color{LightBlue}}
\subsubsectionfont{\color{LightBlue}}
\paragraphfont{\color{LightBlue}}
\subparagraphfont{\color{LightBlue}}
\begin{document}
\definecolor{DarkBlue}{HTML}{4a5a8a} 
\definecolor{LightBlue}{HTML}{4f81bf}
\begin{titlepage}
	\begin{flushleft}
		\vspace{1cm} \Huge  \textbf{Cretaceous Gardens Controller}\\
		\vspace{1cm} \Huge  \textit{Software Requirements Specification}\\
		\vspace{1cm} \Large \textit{SRS Version 1.0}\\
		\vspace{5cm} \LARGE         Team \#3\\ 
		                            29 October 2019
		\vfill       \Huge  \textbf{CS 460 Software Engineering}
	\end{flushleft}
\end{titlepage}
\normalsize \tableofcontents
\pagebreak

\section{Introduction} \label{intro} % Zeke
% project goals and the purpose of this document
	\paragraph{} The purpose of this document is to \textit{specify} the requirements for the
	 development of the Cretaceous Gardens Controller (CGC). The specification is formalized and 
	 diagrammed in order to guide the eventual implementation of the system. Information 
	 encountered in the corresponding \textit{Requirements Definition Document} is reiterated 
	 and restated here where relevant.
	  
	 \paragraph{} After this introduction \footnote{Introduction by Ezequiel Ramos}, 
	 Section \ref{gen} gives an overview of the system. Section \ref{spec} delves into more 
	 detail with subsections \ref{logic} and \ref{inter} that feature a more granular view 
	 of the \textit{Control Logic} and the \textit{External Interfaces}. Section \ref{defs} 
	 provides the definition of technical terms that will be commonly used.
\pagebreak
\section{General Description} \label{gen} %Zeke
	\paragraph{}\textit{This section \footnote{General Description by Ezequiel Ramos and 
	Santiago Cejas} provides a general overview of the whole system. How the system interacts with 
	the hardware interfaces and its basic functionality are introduced here. A description of
	parts to be used in the system and the available functionalities for each type are also 
	provided. Some high level constraints and assumptions for the system will be also be presented.
	It should be noted that a more detailed specification of constraints is covered in its 
	own section.}
	
	\subsection{Product Perspective}
		\paragraph{} 

		\paragraph{} 
		
%		\begin{figure}[H]
%  			\centerline{\includegraphics[scale=.52]{Block_Diagram.png}}
%  			\caption{Block Diagram}
%  			\label{fig:block}
%		\end{figure}
		
		\paragraph{} 
		\paragraph{} 	
		\paragraph{} 

	\subsection{Product Functions}
		\paragraph{} 
		
		\paragraph{} 
		
		\paragraph{} 
		
		\paragraph{} 

	\subsection{User Features}
		\paragraph{} 
		
		\paragraph{} 
		
		\paragraph{} 
	
	\subsection{High Level Constraints}
		\paragraph{}

	\subsection{Assumptions}
	\paragraph{} We assume that the infrastructure is all redundant. The CGC is 
	installed on redundant servers. The network backbone has physical redundant 
	links to appropriate devices like the cameras, the PA speakers, and the 
	electric fence. We will also	program redundancy into the logic. Like the 
	ability to have another car available in case of an emergency or if the car 
	breaks down.
		
	\paragraph{} Another assumption is that messages would be encrypted in order 
	to provide the security needed, so the messages can not be intercepted and 
	modified.

\section{Specific Requirements} \label{spec} % Siri and Anas
% logical diagram
\paragraph{} \textit{Section Introduction}

	\subsection{Interfaces} \label{inter}% Anas
	\paragraph{} The Interfaces \footnote{External Interfaces by Anas Gauba} 
	make up all the pieces that the CGC communicates with. The CGC itself must 
	communicate with everything, 	but a lot of interfaces can function on their own. The 
	car interface is an example of one that needs to be able to function on it's own.
		
	\paragraph{Pay Kiosk}
	\textit{}
	    \subparagraph{Incoming Events}
		\begin{enumerate}
		    \item Register visitor(demographics)/request money.
		    \item Accept money(type)/build token.
		\end{enumerate}
				
	    \subparagraph{Outgoing Events}
		\begin{enumerate}
            \item Activate token(id).
            \item Dispense token(id). 
            \item Dispense change(money, receipt).
            \item Log transaction. 
            \item Report health status to CGC. 
		\end{enumerate}

	\paragraph{Token}
	\textit{}
	    \subparagraph{Incoming Events}
		\begin{enumerate}
            \item Trigger Alarm. 
            \item Return to car(carID). 
		\end{enumerate}
				
	    \subparagraph{Outgoing Events}
		\begin{enumerate}
			\item Report location to GPS Server(gpsID).
		\end{enumerate}

	\paragraph{Car}
	\textit{}
	    \subparagraph{Incoming Events}
		\begin{enumerate}
            \item Read token(tokenID)/Unlock doors or deny access. 
            \item Activate car()[Normal Mode]/Go to south end to pick up visitors.
            \item Activate car()[Emergency Mode]/Go to north end to pick up visitors.
            \item Arrived(Destination)[Normal Mode]/pick up or drop off visitors following the conditioned the protocol.
            \item Arrived(Destination)[Emergency Mode]/pick up or drop off visitors following the conditioned the protocol.
            \item Weight detected.
            \item Change driving mode(modeName).
            \item Activate intercom.
		\end{enumerate}
				
	    \subparagraph{Outgoing Events}
		\begin{enumerate}
		    \item The GPS current location(id).
		    \item Alert visitors(carID).
		    \item Trigger alarm.
		    \item Report health status to CGC.
		\end{enumerate}

    \paragraph{T-Rex Monitor}
	\textit{}
	    \subparagraph{Incoming Events}
		\begin{enumerate}
			\item Inject tranquilizer. 
		\end{enumerate}
				
	    \subparagraph{Outgoing Events}
		\begin{enumerate}
			\item Report T-Rex health.
			\item Report health status to CGC.
			\item Report location to GPS Server(gpsID).
		\end{enumerate}

	\paragraph{Camera Network}
	\textit{}
	    \subparagraph{Incoming Events}
		\begin{enumerate}
			\item Delete recording(cameraID, date range).
			\item Activate recording(cameraID).
			\item Monitor streaming(cameraID).
		\end{enumerate}
		
	    \subparagraph{Outgoing Events}
		\begin{enumerate}
			\item Camera outage(cameraID).
			\item Report health status to CGC. 
		\end{enumerate}

	\paragraph{Electric Fence}
	\textit{}
	    \subparagraph{Incoming Events}
		\begin{enumerate}
			\item Null.
		\end{enumerate}
				
	    \subparagraph{Outgoing Events}
		\begin{enumerate}
		    \item Electricity distortion/trigger an emergency mode.
		    \item Report health status to CGC. 
		\end{enumerate}

	\paragraph{Global Alarm System}
	\textit{}
	    \subparagraph{Incoming Events}
		\begin{enumerate}
			\item Trigger alarms[Emergency Mode]/play emergency alarm sound.
			\item Trigger alarms[Normal Mode]/play Public Service Annoucement (PSA).
			\item Disable alarms.
		\end{enumerate}
				
	    \subparagraph{Outgoing Events}
		\begin{enumerate}
			\item Report health status to CGC.
		\end{enumerate}

	\paragraph{CGC Station}
	\textit{}
	    \subparagraph{Incoming Events}
		\begin{enumerate}
			\item Review health status of all the associated devices.
		\end{enumerate}
				
	    \subparagraph{Outgoing Events}
		\begin{enumerate}
			\item Activate tranquilizer.
			\item Deactivate emergency mode.
			\item Activate intercom. 
		\end{enumerate}

	\paragraph{GPS Server}
	\textit{}
	    \subparagraph{Incoming Events}
		\begin{enumerate}
			\item Track location(gpsID).
		\end{enumerate}
				
	    \subparagraph{Outgoing Events}
		\begin{enumerate}
			\item Report location(gpsID).
		\end{enumerate}						
		
    \subsection{Control Logic} \label{logic}%siri
	\paragraph{} \textit{ \footnote{Control Logic by Siri Khalsa}}

%    \begin{figure}[H]
%  		\centerline{\includegraphics[scale=1]{ECSModes.jpg}}
%  		\caption{Elevator Control System Normal Function Model }
%  		\label{fig:normal}
%	\end{figure}
%
%	\begin{figure}[H]
%  	    \centerline{\includegraphics[scale=1]{ECS-Normal.jpg}}
%  		\caption{Elevator Control System Normal Function Model }
%  		\label{fig:normal}
%	\end{figure}
%	
%	\begin{figure}[H]
%  		\centerline{\includegraphics[scale=1]{EmergencyPhase1.jpg}}
%  		\caption{Elevator Control System Normal Function Model }
%  		\label{fig:emerg}
%	\end{figure}

\section{Design Constraints} \label{cons} %
% error handling, exceptions, hazards, 
\paragraph{} \textit{Section Intro}

	\subsection{Client}
	\begin{itemize}
		\item 
	\end{itemize}

	\subsection{Safety}
	\begin{itemize}
		\item 
	\end{itemize}

	\subsection{Regulations}
	\begin{itemize}
		\item 
	\end{itemize}

	\subsection{Security}
	\begin{itemize}
		\item 
	\end{itemize}

\section{Sample Use Cases}
\paragraph{} \textit{intro saying categorized by actor with diagrams blah blah blah these uses are just samples}

%%%%% Anas %%%%%
    \subsection{Bookkeeper}
    \textit{brief description of actor}
    \par\noindent\rule{\textwidth}{0.4pt}    
    \begin{itemize}
        \item[]\textbf{Use Case:}                                
            *DoSomething (the name of the action to be executed by actor)*

        \item[]\textbf{Primary Actor:}
            *ThatWhichWishesToDoSomething (name of actor)*

        \item[]\textbf{Goal in Context:}
            *that which is to be accomplished by the action*

        \item[]\textbf{Preconditions:}
            *states of the actor and system prior to the action*

        \item[]\textbf{Trigger:}
            *that which initiates the action (e.g. actor breaks out of enclosure)*

        \item[]\textbf{Scenario:}
            *sequence of actions from trigger to goal*
            \begin{enumerate}
                \item first event
                \item second event
                \item ...
            \end{enumerate}

        \item[]\textbf{Exceptions:}
            *edge cases, potential hazards, errors, etc*
            \begin{itemize}
                \item[] some exception
                \item[] some other exception                
                \item[] ...
            \end{itemize}

        \item[]\textbf{Priority:}
            *level of implementation importance (e.g. correct change is a must)*

        \item[]\textbf{When Available:}
            *when or during which interval of time the action is to supported by the system*

        \item[]\textbf{Frequency of Use:}
            *number of uses per unit of time (e.g. annually, billions per second, etc)*

        \item[]\textbf{Channel to Primary Actor:}
            *means through which the system interacts with actor*

        \item[]\textbf{Channels to Secondary Actors:}
            *means through which the primary and secondary actors interact*
            \begin{itemize}
                \item[] some channel
                \item[] some other channel
                \item[] ...
            \end{itemize}
        \item[]\textbf{Secondary Actors:}
            *intermediary or auxillary actors required to complete the goal*

        \item[]\textbf{Open Issues:}
            *itemization of current problems with any of the above*
            \begin{itemize}
                \item[] some issue
                \item[] some other issue
                \item[] ...
            \end{itemize}
    \end{itemize}
    
    \subsection{CGC Station Operator}
    \textit{brief description of actor}
    \par\noindent\rule{\textwidth}{0.4pt}    
    \begin{itemize}
        \item[]\textbf{Use Case:}                                
            *DoSomething (the name of the action to be executed by actor)*

        \item[]\textbf{Primary Actor:}
            *ThatWhichWishesToDoSomething (name of actor)*

        \item[]\textbf{Goal in Context:}
            *that which is to be accomplished by the action*

        \item[]\textbf{Preconditions:}
            *states of the actor and system prior to the action*

        \item[]\textbf{Trigger:}
            *that which initiates the action (e.g. actor breaks out of enclosure)*

        \item[]\textbf{Scenario:}
            *sequence of actions from trigger to goal*
            \begin{enumerate}
                \item first event
                \item second event
                \item ...
            \end{enumerate}

        \item[]\textbf{Exceptions:}
            *edge cases, potential hazards, errors, etc*
            \begin{itemize}
                \item[] some exception
                \item[] some other exception                
                \item[] ...
            \end{itemize}

        \item[]\textbf{Priority:}
            *level of implementation importance (e.g. correct change is a must)*

        \item[]\textbf{When Available:}
            *when or during which interval of time the action is to supported by the system*

        \item[]\textbf{Frequency of Use:}
            *number of uses per unit of time (e.g. annually, billions per second, etc)*

        \item[]\textbf{Channel to Primary Actor:}
            *means through which the system interacts with actor*

        \item[]\textbf{Channels to Secondary Actors:}
            *means through which the primary and secondary actors interact*
            \begin{itemize}
                \item[] some channel
                \item[] some other channel
                \item[] ...
            \end{itemize}
        \item[]\textbf{Secondary Actors:}
            *intermediary or auxillary actors required to complete the goal*

        \item[]\textbf{Open Issues:}
            *itemization of current problems with any of the above*
            \begin{itemize}
                \item[] some issue
                \item[] some other issue
                \item[] ...
            \end{itemize}
    \end{itemize}
     
    \subsection{Emergency Personnel}
    \textit{brief description of actor}
    \par\noindent\rule{\textwidth}{0.4pt}    
    \begin{itemize}
        \item[]\textbf{Use Case:}                                
            *DoSomething (the name of the action to be executed by actor)*

        \item[]\textbf{Primary Actor:}
            *ThatWhichWishesToDoSomething (name of actor)*

        \item[]\textbf{Goal in Context:}
            *that which is to be accomplished by the action*

        \item[]\textbf{Preconditions:}
            *states of the actor and system prior to the action*

        \item[]\textbf{Trigger:}
            *that which initiates the action (e.g. actor breaks out of enclosure)*

        \item[]\textbf{Scenario:}
            *sequence of actions from trigger to goal*
            \begin{enumerate}
                \item first event
                \item second event
                \item ...
            \end{enumerate}

        \item[]\textbf{Exceptions:}
            *edge cases, potential hazards, errors, etc*
            \begin{itemize}
                \item[] some exception
                \item[] some other exception                
                \item[] ...
            \end{itemize}

        \item[]\textbf{Priority:}
            *level of implementation importance (e.g. correct change is a must)*

        \item[]\textbf{When Available:}
            *when or during which interval of time the action is to supported by the system*

        \item[]\textbf{Frequency of Use:}
            *number of uses per unit of time (e.g. annually, billions per second, etc)*

        \item[]\textbf{Channel to Primary Actor:}
            *means through which the system interacts with actor*

        \item[]\textbf{Channels to Secondary Actors:}
            *means through which the primary and secondary actors interact*
            \begin{itemize}
                \item[] some channel
                \item[] some other channel
                \item[] ...
            \end{itemize}
        \item[]\textbf{Secondary Actors:}
            *intermediary or auxillary actors required to complete the goal*

        \item[]\textbf{Open Issues:}
            *itemization of current problems with any of the above*
            \begin{itemize}
                \item[] some issue
                \item[] some other issue
                \item[] ...
            \end{itemize}
    \end{itemize}
     
%%%%% Matt %%%%%
    \subsection{Enclosure Maintenance Personnel}
    \textit{brief description of actor}
    \par\noindent\rule{\textwidth}{0.4pt}    
    \begin{itemize}
        \item[]\textbf{Use Case:}                                
            *DoSomething (the name of the action to be executed by actor)*

        \item[]\textbf{Primary Actor:}
            *ThatWhichWishesToDoSomething (name of actor)*

        \item[]\textbf{Goal in Context:}
            *that which is to be accomplished by the action*

        \item[]\textbf{Preconditions:}
            *states of the actor and system prior to the action*

        \item[]\textbf{Trigger:}
            *that which initiates the action (e.g. actor breaks out of enclosure)*

        \item[]\textbf{Scenario:}
            *sequence of actions from trigger to goal*
            \begin{enumerate}
                \item first event
                \item second event
                \item ...
            \end{enumerate}

        \item[]\textbf{Exceptions:}
            *edge cases, potential hazards, errors, etc*
            \begin{itemize}
                \item[] some exception
                \item[] some other exception                
                \item[] ...
            \end{itemize}

        \item[]\textbf{Priority:}
            *level of implementation importance (e.g. correct change is a must)*

        \item[]\textbf{When Available:}
            *when or during which interval of time the action is to supported by the system*

        \item[]\textbf{Frequency of Use:}
            *number of uses per unit of time (e.g. annually, billions per second, etc)*

        \item[]\textbf{Channel to Primary Actor:}
            *means through which the system interacts with actor*

        \item[]\textbf{Channels to Secondary Actors:}
            *means through which the primary and secondary actors interact*
            \begin{itemize}
                \item[] some channel
                \item[] some other channel
                \item[] ...
            \end{itemize}
        \item[]\textbf{Secondary Actors:}
            *intermediary or auxillary actors required to complete the goal*

        \item[]\textbf{Open Issues:}
            *itemization of current problems with any of the above*
            \begin{itemize}
                \item[] some issue
                \item[] some other issue
                \item[] ...
            \end{itemize}
    \end{itemize}
     
    \subsection{Guest}
    \textit{brief description of actor}
    \par\noindent\rule{\textwidth}{0.4pt}    
    \begin{itemize} %ViewTRex
        \item[]\textbf{Use Case:}                                
            *DoSomething (the name of the action to be executed by actor)*

        \item[]\textbf{Primary Actor:}
            *ThatWhichWishesToDoSomething (name of actor)*

        \item[]\textbf{Goal in Context:}
            *that which is to be accomplished by the action*

        \item[]\textbf{Preconditions:}
            *states of the actor and system prior to the action*

        \item[]\textbf{Trigger:}
            *that which initiates the action (e.g. actor breaks out of enclosure)*

        \item[]\textbf{Scenario:}
            *sequence of actions from trigger to goal*
            \begin{enumerate}
                \item first event
                \item second event
                \item ...
            \end{enumerate}

        \item[]\textbf{Exceptions:}
            *edge cases, potential hazards, errors, etc*
            \begin{itemize}
                \item[] some exception
                \item[] some other exception                
                \item[] ...
            \end{itemize}

        \item[]\textbf{Priority:}
            *level of implementation importance (e.g. correct change is a must)*

        \item[]\textbf{When Available:}
            *when or during which interval of time the action is to supported by the system*

        \item[]\textbf{Frequency of Use:}
            *number of uses per unit of time (e.g. annually, billions per second, etc)*

        \item[]\textbf{Channel to Primary Actor:}
            *means through which the system interacts with actor*

        \item[]\textbf{Channels to Secondary Actors:}
            *means through which the primary and secondary actors interact*
            \begin{itemize}
                \item[] some channel
                \item[] some other channel
                \item[] ...
            \end{itemize}
        \item[]\textbf{Secondary Actors:}
            *intermediary or auxillary actors required to complete the goal*

        \item[]\textbf{Open Issues:}
            *itemization of current problems with any of the above*
            \begin{itemize}
                \item[] some issue
                \item[] some other issue
                \item[] ...
            \end{itemize}
    \end{itemize}
     
    \par\noindent\rule{\textwidth}{0.4pt}    
    \begin{itemize} %LeaveResort
        \item[]\textbf{Use Case:}                                
            *DoSomething (the name of the action to be executed by actor)*

        \item[]\textbf{Primary Actor:}
            *ThatWhichWishesToDoSomething (name of actor)*

        \item[]\textbf{Goal in Context:}
            *that which is to be accomplished by the action*

        \item[]\textbf{Preconditions:}
            *states of the actor and system prior to the action*

        \item[]\textbf{Trigger:}
            *that which initiates the action (e.g. actor breaks out of enclosure)*

        \item[]\textbf{Scenario:}
            *sequence of actions from trigger to goal*
            \begin{enumerate}
                \item first event
                \item second event
                \item ...
            \end{enumerate}

        \item[]\textbf{Exceptions:}
            *edge cases, potential hazards, errors, etc*
            \begin{itemize}
                \item[] some exception
                \item[] some other exception                
                \item[] ...
            \end{itemize}

        \item[]\textbf{Priority:}
            *level of implementation importance (e.g. correct change is a must)*

        \item[]\textbf{When Available:}
            *when or during which interval of time the action is to supported by the system*

        \item[]\textbf{Frequency of Use:}
            *number of uses per unit of time (e.g. annually, billions per second, etc)*

        \item[]\textbf{Channel to Primary Actor:}
            *means through which the system interacts with actor*

        \item[]\textbf{Channels to Secondary Actors:}
            *means through which the primary and secondary actors interact*
            \begin{itemize}
                \item[] some channel
                \item[] some other channel
                \item[] ...
            \end{itemize}
        \item[]\textbf{Secondary Actors:}
            *intermediary or auxillary actors required to complete the goal*

        \item[]\textbf{Open Issues:}
            *itemization of current problems with any of the above*
            \begin{itemize}
                \item[] some issue
                \item[] some other issue
                \item[] ...
            \end{itemize}
    \end{itemize}
     
    \par\noindent\rule{\textwidth}{0.4pt}    
    \begin{itemize} %PurchaseToken
        \item[]\textbf{Use Case:}                                
            *DoSomething (the name of the action to be executed by actor)*

        \item[]\textbf{Primary Actor:}
            *ThatWhichWishesToDoSomething (name of actor)*

        \item[]\textbf{Goal in Context:}
            *that which is to be accomplished by the action*

        \item[]\textbf{Preconditions:}
            *states of the actor and system prior to the action*

        \item[]\textbf{Trigger:}
            *that which initiates the action (e.g. actor breaks out of enclosure)*

        \item[]\textbf{Scenario:}
            *sequence of actions from trigger to goal*
            \begin{enumerate}
                \item first event
                \item second event
                \item ...
            \end{enumerate}

        \item[]\textbf{Exceptions:}
            *edge cases, potential hazards, errors, etc*
            \begin{itemize}
                \item[] some exception
                \item[] some other exception                
                \item[] ...
            \end{itemize}

        \item[]\textbf{Priority:}
            *level of implementation importance (e.g. correct change is a must)*

        \item[]\textbf{When Available:}
            *when or during which interval of time the action is to supported by the system*

        \item[]\textbf{Frequency of Use:}
            *number of uses per unit of time (e.g. annually, billions per second, etc)*

        \item[]\textbf{Channel to Primary Actor:}
            *means through which the system interacts with actor*

        \item[]\textbf{Channels to Secondary Actors:}
            *means through which the primary and secondary actors interact*
            \begin{itemize}
                \item[] some channel
                \item[] some other channel
                \item[] ...
            \end{itemize}
        \item[]\textbf{Secondary Actors:}
            *intermediary or auxillary actors required to complete the goal*

        \item[]\textbf{Open Issues:}
            *itemization of current problems with any of the above*
            \begin{itemize}
                \item[] some issue
                \item[] some other issue
                \item[] ...
            \end{itemize}
    \end{itemize}

    \par\noindent\rule{\textwidth}{0.4pt}    
    \begin{itemize} %Evacuate
        \item[]\textbf{Use Case:}                                
            *DoSomething (the name of the action to be executed by actor)*

        \item[]\textbf{Primary Actor:}
            *ThatWhichWishesToDoSomething (name of actor)*

        \item[]\textbf{Goal in Context:}
            *that which is to be accomplished by the action*

        \item[]\textbf{Preconditions:}
            *states of the actor and system prior to the action*

        \item[]\textbf{Trigger:}
            *that which initiates the action (e.g. actor breaks out of enclosure)*

        \item[]\textbf{Scenario:}
            *sequence of actions from trigger to goal*
            \begin{enumerate}
                \item first event
                \item second event
                \item ...
            \end{enumerate}

        \item[]\textbf{Exceptions:}
            *edge cases, potential hazards, errors, etc*
            \begin{itemize}
                \item[] some exception
                \item[] some other exception                
                \item[] ...
            \end{itemize}

        \item[]\textbf{Priority:}
            *level of implementation importance (e.g. correct change is a must)*

        \item[]\textbf{When Available:}
            *when or during which interval of time the action is to supported by the system*

        \item[]\textbf{Frequency of Use:}
            *number of uses per unit of time (e.g. annually, billions per second, etc)*

        \item[]\textbf{Channel to Primary Actor:}
            *means through which the system interacts with actor*

        \item[]\textbf{Channels to Secondary Actors:}
            *means through which the primary and secondary actors interact*
            \begin{itemize}
                \item[] some channel
                \item[] some other channel
                \item[] ...
            \end{itemize}
        \item[]\textbf{Secondary Actors:}
            *intermediary or auxillary actors required to complete the goal*

        \item[]\textbf{Open Issues:}
            *itemization of current problems with any of the above*
            \begin{itemize}
                \item[] some issue
                \item[] some other issue
                \item[] ...
            \end{itemize}
    \end{itemize}
    
    % another use case
    
    % another use case
    
    % ...
    
    \subsection{Guest Vehicle}
    \textit{brief description of actor}
    \par\noindent\rule{\textwidth}{0.4pt}    
    \begin{itemize} %ShuttleGuestsToExhibit
        \item[]\textbf{Use Case:}                                
            *DoSomething (the name of the action to be executed by actor)*

        \item[]\textbf{Primary Actor:}
            *ThatWhichWishesToDoSomething (name of actor)*

        \item[]\textbf{Goal in Context:}
            *that which is to be accomplished by the action*

        \item[]\textbf{Preconditions:}
            *states of the actor and system prior to the action*

        \item[]\textbf{Trigger:}
            *that which initiates the action (e.g. actor breaks out of enclosure)*

        \item[]\textbf{Scenario:}
            *sequence of actions from trigger to goal*
            \begin{enumerate}
                \item first event
                \item second event
                \item ...
            \end{enumerate}

        \item[]\textbf{Exceptions:}
            *edge cases, potential hazards, errors, etc*
            \begin{itemize}
                \item[] some exception
                \item[] some other exception                
                \item[] ...
            \end{itemize}

        \item[]\textbf{Priority:}
            *level of implementation importance (e.g. correct change is a must)*

        \item[]\textbf{When Available:}
            *when or during which interval of time the action is to supported by the system*

        \item[]\textbf{Frequency of Use:}
            *number of uses per unit of time (e.g. annually, billions per second, etc)*

        \item[]\textbf{Channel to Primary Actor:}
            *means through which the system interacts with actor*

        \item[]\textbf{Channels to Secondary Actors:}
            *means through which the primary and secondary actors interact*
            \begin{itemize}
                \item[] some channel
                \item[] some other channel
                \item[] ...
            \end{itemize}
        \item[]\textbf{Secondary Actors:}
            *intermediary or auxillary actors required to complete the goal*

        \item[]\textbf{Open Issues:}
            *itemization of current problems with any of the above*
            \begin{itemize}
                \item[] some issue
                \item[] some other issue
                \item[] ...
            \end{itemize}
    \end{itemize}

    \par\noindent\rule{\textwidth}{0.4pt}    
    \begin{itemize} %ShuttleGuestsFromExhibit
        \item[]\textbf{Use Case:}                                
            *DoSomething (the name of the action to be executed by actor)*

        \item[]\textbf{Primary Actor:}
            *ThatWhichWishesToDoSomething (name of actor)*

        \item[]\textbf{Goal in Context:}
            *that which is to be accomplished by the action*

        \item[]\textbf{Preconditions:}
            *states of the actor and system prior to the action*

        \item[]\textbf{Trigger:}
            *that which initiates the action (e.g. actor breaks out of enclosure)*

        \item[]\textbf{Scenario:}
            *sequence of actions from trigger to goal*
            \begin{enumerate}
                \item first event
                \item second event
                \item ...
            \end{enumerate}

        \item[]\textbf{Exceptions:}
            *edge cases, potential hazards, errors, etc*
            \begin{itemize}
                \item[] some exception
                \item[] some other exception                
                \item[] ...
            \end{itemize}

        \item[]\textbf{Priority:}
            *level of implementation importance (e.g. correct change is a must)*

        \item[]\textbf{When Available:}
            *when or during which interval of time the action is to supported by the system*

        \item[]\textbf{Frequency of Use:}
            *number of uses per unit of time (e.g. annually, billions per second, etc)*

        \item[]\textbf{Channel to Primary Actor:}
            *means through which the system interacts with actor*

        \item[]\textbf{Channels to Secondary Actors:}
            *means through which the primary and secondary actors interact*
            \begin{itemize}
                \item[] some channel
                \item[] some other channel
                \item[] ...
            \end{itemize}
        \item[]\textbf{Secondary Actors:}
            *intermediary or auxillary actors required to complete the goal*

        \item[]\textbf{Open Issues:}
            *itemization of current problems with any of the above*
            \begin{itemize}
                \item[] some issue
                \item[] some other issue
                \item[] ...
            \end{itemize}
    \end{itemize}
    
    % another use case
    
    % another use case
    
    % ...
%%%%% Santi %%%%%
    \subsection{Network Maintenance Personnel}
    \textit{brief description of actor}
    \par\noindent\rule{\textwidth}{0.4pt}    
    \begin{itemize}
        \item[]\textbf{Use Case:}                                
            *DoSomething (the name of the action to be executed by actor)*

        \item[]\textbf{Primary Actor:}
            *ThatWhichWishesToDoSomething (name of actor)*

        \item[]\textbf{Goal in Context:}
            *that which is to be accomplished by the action*

        \item[]\textbf{Preconditions:}
            *states of the actor and system prior to the action*

        \item[]\textbf{Trigger:}
            *that which initiates the action (e.g. actor breaks out of enclosure)*

        \item[]\textbf{Scenario:}
            *sequence of actions from trigger to goal*
            \begin{enumerate}
                \item first event
                \item second event
                \item ...
            \end{enumerate}

        \item[]\textbf{Exceptions:}
            *edge cases, potential hazards, errors, etc*
            \begin{itemize}
                \item[] some exception
                \item[] some other exception                
                \item[] ...
            \end{itemize}

        \item[]\textbf{Priority:}
            *level of implementation importance (e.g. correct change is a must)*

        \item[]\textbf{When Available:}
            *when or during which interval of time the action is to supported by the system*

        \item[]\textbf{Frequency of Use:}
            *number of uses per unit of time (e.g. annually, billions per second, etc)*

        \item[]\textbf{Channel to Primary Actor:}
            *means through which the system interacts with actor*

        \item[]\textbf{Channels to Secondary Actors:}
            *means through which the primary and secondary actors interact*
            \begin{itemize}
                \item[] some channel
                \item[] some other channel
                \item[] ...
            \end{itemize}
        \item[]\textbf{Secondary Actors:}
            *intermediary or auxillary actors required to complete the goal*

        \item[]\textbf{Open Issues:}
            *itemization of current problems with any of the above*
            \begin{itemize}
                \item[] some issue
                \item[] some other issue
                \item[] ...
            \end{itemize}
    \end{itemize}
    
    % another use case
    
    % another use case
    
    % ...
    
    \subsection{Patrol Vehicle}
    \textit{brief description of actor}
    \par\noindent\rule{\textwidth}{0.4pt}    
    \begin{itemize}
        \item[]\textbf{Use Case:}                                
            *DoSomething (the name of the action to be executed by actor)*

        \item[]\textbf{Primary Actor:}
            *ThatWhichWishesToDoSomething (name of actor)*

        \item[]\textbf{Goal in Context:}
            *that which is to be accomplished by the action*

        \item[]\textbf{Preconditions:}
            *states of the actor and system prior to the action*

        \item[]\textbf{Trigger:}
            *that which initiates the action (e.g. actor breaks out of enclosure)*

        \item[]\textbf{Scenario:}
            *sequence of actions from trigger to goal*
            \begin{enumerate}
                \item first event
                \item second event
                \item ...
            \end{enumerate}

        \item[]\textbf{Exceptions:}
            *edge cases, potential hazards, errors, etc*
            \begin{itemize}
                \item[] some exception
                \item[] some other exception                
                \item[] ...
            \end{itemize}

        \item[]\textbf{Priority:}
            *level of implementation importance (e.g. correct change is a must)*

        \item[]\textbf{When Available:}
            *when or during which interval of time the action is to supported by the system*

        \item[]\textbf{Frequency of Use:}
            *number of uses per unit of time (e.g. annually, billions per second, etc)*

        \item[]\textbf{Channel to Primary Actor:}
            *means through which the system interacts with actor*

        \item[]\textbf{Channels to Secondary Actors:}
            *means through which the primary and secondary actors interact*
            \begin{itemize}
                \item[] some channel
                \item[] some other channel
                \item[] ...
            \end{itemize}
        \item[]\textbf{Secondary Actors:}
            *intermediary or auxillary actors required to complete the goal*

        \item[]\textbf{Open Issues:}
            *itemization of current problems with any of the above*
            \begin{itemize}
                \item[] some issue
                \item[] some other issue
                \item[] ...
            \end{itemize}
    \end{itemize}
    
    % another use case
    
    % another use case
    
    % ...
    
    \subsection{Sales department}
    \textit{brief description of actor}
    \par\noindent\rule{\textwidth}{0.4pt}    
    \begin{itemize}
        \item[]\textbf{Use Case:}                                
            *DoSomething (the name of the action to be executed by actor)*

        \item[]\textbf{Primary Actor:}
            *ThatWhichWishesToDoSomething (name of actor)*

        \item[]\textbf{Goal in Context:}
            *that which is to be accomplished by the action*

        \item[]\textbf{Preconditions:}
            *states of the actor and system prior to the action*

        \item[]\textbf{Trigger:}
            *that which initiates the action (e.g. actor breaks out of enclosure)*

        \item[]\textbf{Scenario:}
            *sequence of actions from trigger to goal*
            \begin{enumerate}
                \item first event
                \item second event
                \item ...
            \end{enumerate}

        \item[]\textbf{Exceptions:}
            *edge cases, potential hazards, errors, etc*
            \begin{itemize}
                \item[] some exception
                \item[] some other exception                
                \item[] ...
            \end{itemize}

        \item[]\textbf{Priority:}
            *level of implementation importance (e.g. correct change is a must)*

        \item[]\textbf{When Available:}
            *when or during which interval of time the action is to supported by the system*

        \item[]\textbf{Frequency of Use:}
            *number of uses per unit of time (e.g. annually, billions per second, etc)*

        \item[]\textbf{Channel to Primary Actor:}
            *means through which the system interacts with actor*

        \item[]\textbf{Channels to Secondary Actors:}
            *means through which the primary and secondary actors interact*
            \begin{itemize}
                \item[] some channel
                \item[] some other channel
                \item[] ...
            \end{itemize}
        \item[]\textbf{Secondary Actors:}
            *intermediary or auxillary actors required to complete the goal*

        \item[]\textbf{Open Issues:}
            *itemization of current problems with any of the above*
            \begin{itemize}
                \item[] some issue
                \item[] some other issue
                \item[] ...
            \end{itemize}
    \end{itemize}
    
    % another use case
    
    % another use case
    
    % ...

%%%%% Siri %%%%%
    \subsection{System Administrator}
    \textit{brief description of actor}
    \par\noindent\rule{\textwidth}{0.4pt}    
    \begin{itemize}
        \item[]\textbf{Use Case:}                                
            *DoSomething (the name of the action to be executed by actor)*

        \item[]\textbf{Primary Actor:}
            *ThatWhichWishesToDoSomething (name of actor)*

        \item[]\textbf{Goal in Context:}
            *that which is to be accomplished by the action*

        \item[]\textbf{Preconditions:}
            *states of the actor and system prior to the action*

        \item[]\textbf{Trigger:}
            *that which initiates the action (e.g. actor breaks out of enclosure)*

        \item[]\textbf{Scenario:}
            *sequence of actions from trigger to goal*
            \begin{enumerate}
                \item first event
                \item second event
                \item ...
            \end{enumerate}

        \item[]\textbf{Exceptions:}
            *edge cases, potential hazards, errors, etc*
            \begin{itemize}
                \item[] some exception
                \item[] some other exception                
                \item[] ...
            \end{itemize}

        \item[]\textbf{Priority:}
            *level of implementation importance (e.g. correct change is a must)*

        \item[]\textbf{When Available:}
            *when or during which interval of time the action is to supported by the system*

        \item[]\textbf{Frequency of Use:}
            *number of uses per unit of time (e.g. annually, billions per second, etc)*

        \item[]\textbf{Channel to Primary Actor:}
            *means through which the system interacts with actor*

        \item[]\textbf{Channels to Secondary Actors:}
            *means through which the primary and secondary actors interact*
            \begin{itemize}
                \item[] some channel
                \item[] some other channel
                \item[] ...
            \end{itemize}
        \item[]\textbf{Secondary Actors:}
            *intermediary or auxillary actors required to complete the goal*

        \item[]\textbf{Open Issues:}
            *itemization of current problems with any of the above*
            \begin{itemize}
                \item[] some issue
                \item[] some other issue
                \item[] ...
            \end{itemize}
    \end{itemize}
    
    % another use case
    
    % another use case
    
    % ...
    
    \subsection{System Auditor}
    \textit{brief description of actor}
    \par\noindent\rule{\textwidth}{0.4pt}    
    \begin{itemize}
        \item[]\textbf{Use Case:}                                
            *DoSomething (the name of the action to be executed by actor)*

        \item[]\textbf{Primary Actor:}
            *ThatWhichWishesToDoSomething (name of actor)*

        \item[]\textbf{Goal in Context:}
            *that which is to be accomplished by the action*

        \item[]\textbf{Preconditions:}
            *states of the actor and system prior to the action*

        \item[]\textbf{Trigger:}
            *that which initiates the action (e.g. actor breaks out of enclosure)*

        \item[]\textbf{Scenario:}
            *sequence of actions from trigger to goal*
            \begin{enumerate}
                \item first event
                \item second event
                \item ...
            \end{enumerate}

        \item[]\textbf{Exceptions:}
            *edge cases, potential hazards, errors, etc*
            \begin{itemize}
                \item[] some exception
                \item[] some other exception                
                \item[] ...
            \end{itemize}

        \item[]\textbf{Priority:}
            *level of implementation importance (e.g. correct change is a must)*

        \item[]\textbf{When Available:}
            *when or during which interval of time the action is to supported by the system*

        \item[]\textbf{Frequency of Use:}
            *number of uses per unit of time (e.g. annually, billions per second, etc)*

        \item[]\textbf{Channel to Primary Actor:}
            *means through which the system interacts with actor*

        \item[]\textbf{Channels to Secondary Actors:}
            *means through which the primary and secondary actors interact*
            \begin{itemize}
                \item[] some channel
                \item[] some other channel
                \item[] ...
            \end{itemize}
        \item[]\textbf{Secondary Actors:}
            *intermediary or auxillary actors required to complete the goal*

        \item[]\textbf{Open Issues:}
            *itemization of current problems with any of the above*
            \begin{itemize}
                \item[] some issue
                \item[] some other issue
                \item[] ...
            \end{itemize}
    \end{itemize}
    
    % another use case
    
    % another use case
    
    % ...
    
    
    \subsection{System Technician}
    \textit{brief description of actor}
    \par\noindent\rule{\textwidth}{0.4pt}    
    \begin{itemize}
        \item[]\textbf{Use Case:}                                
            *DoSomething (the name of the action to be executed by actor)*

        \item[]\textbf{Primary Actor:}
            *ThatWhichWishesToDoSomething (name of actor)*

        \item[]\textbf{Goal in Context:}
            *that which is to be accomplished by the action*

        \item[]\textbf{Preconditions:}
            *states of the actor and system prior to the action*

        \item[]\textbf{Trigger:}
            *that which initiates the action (e.g. actor breaks out of enclosure)*

        \item[]\textbf{Scenario:}
            *sequence of actions from trigger to goal*
            \begin{enumerate}
                \item first event
                \item second event
                \item ...
            \end{enumerate}

        \item[]\textbf{Exceptions:}
            *edge cases, potential hazards, errors, etc*
            \begin{itemize}
                \item[] some exception
                \item[] some other exception                
                \item[] ...
            \end{itemize}

        \item[]\textbf{Priority:}
            *level of implementation importance (e.g. correct change is a must)*

        \item[]\textbf{When Available:}
            *when or during which interval of time the action is to supported by the system*

        \item[]\textbf{Frequency of Use:}
            *number of uses per unit of time (e.g. annually, billions per second, etc)*

        \item[]\textbf{Channel to Primary Actor:}
            *means through which the system interacts with actor*

        \item[]\textbf{Channels to Secondary Actors:}
            *means through which the primary and secondary actors interact*
            \begin{itemize}
                \item[] some channel
                \item[] some other channel
                \item[] ...
            \end{itemize}
        \item[]\textbf{Secondary Actors:}
            *intermediary or auxillary actors required to complete the goal*

        \item[]\textbf{Open Issues:}
            *itemization of current problems with any of the above*
            \begin{itemize}
                \item[] some issue
                \item[] some other issue
                \item[] ...
            \end{itemize}
    \end{itemize}

    % another use case
    
    % another use case
    
    % ...
    

%%%%% Zeke %%%%%
    \subsection{Tyrannosaurus Rex}
    \textit{It may be argued that this is not a legitimate actor, but despite its unconscious interaction
    with the system, the T.Rex can act on the system in a variety of - possibly unpredictable - ways.}
    \par\noindent\rule{\textwidth}{0.4pt}    
    \begin{itemize}
        \item[]\textbf{Use Case:}
            LeaveEnclosure

        \item[]\textbf{Primary Actor:}
            T.Rex

        \item[]\textbf{Goal in Context:}
            To get somewhere that happens to be outside the enclosure.

        \item[]\textbf{Preconditions:}
            Actor is not sedated, the system is not in maintenance mode 
            nor emergency mode, and all components are functioning properly.

        \item[]\textbf{Trigger:}
            The T.Rex sees or smells something outside the enclosure.

        \item[]\textbf{Scenario:}                    
            \begin{enumerate}
                \item The actor looks through the enclosure, 
                toward an imagined near-future destination beyond
                the enclosure. \label{beginTRLeave}
                \item The actor walks toward the target destination.
                \item The actor is impeded by the electric fence. \label{impedeTRLeave}
                \item The actor becomes fearful.
                \begin{enumerate}
                    \item The actor retreats from the fence.
                    \item[OR]
                    \item The actor attacks the fence. 
                \end{enumerate}
                \item The electric fence increases its voltage.
                \item The scenario may repeat from either 
                act \ref{beginTRLeave}, from act \ref{impedeTRLeave}, 
                or continues such that:
                \begin{enumerate}
                    \item the actor is sedated to prevent further damage
                    to self or enclosure, and maintenance mode is triggered
                    \item[OR]
                    \item the enclosure is breached, the actor 
                    heads toward the target destination, and emergency
                    mode is triggered.
                    \item[OR]
                    \item the actor relinquishes the desire to head
                    toward the target destination, no significant damage
                    is incurred, and the normal mode of operation continues.
                    \end{enumerate} 
            \end{enumerate}

        \item[]\textbf{Exceptions:}
            \begin{itemize}
                \item[] Actor Perishes.
            \end{itemize}

        \item[]\textbf{Priority:}
            Essential, must be implemented.

        \item[]\textbf{When Available:}
            At random.

        \item[]\textbf{Frequency of Use:}
            Preferably never, but less likely with time (ideally)

        \item[]\textbf{Channels to Primary Actor:}
            \begin{itemize}
                \item[] Electric Enclosure Panel
                \item[] T.Rex Monitor
            \end{itemize}
            
        \item[]\textbf{Secondary Actors:}
            CGC Station Operator, Global Alarm System
            
        \item[]\textbf{Channels to Secondary Actors:}  
            \begin{itemize}
                \item[] CGC Station Operator: Camera Network, T.Rex Monitor
                \item[] Global Alarm System: Electric Enclosure Panel
            \end{itemize}

        \item[]\textbf{Open Issues:}
            \begin{itemize}
                \item[] None known.
            \end{itemize}
    \end{itemize}
    
    % another use case
    
    % another use case
    
    % ...
    
    \subsection{Veterinarian}
    \textit{The veterinarian role includes uses such as routine checkups or medical treatment for the T.Rex.}
    \par\noindent\rule{\textwidth}{0.4pt}    
    \begin{itemize}
        \item[]\textbf{Use Case:}                                
            RoutineCheckup

        \item[]\textbf{Primary Actor:}
            Veterinarian

        \item[]\textbf{Goal in Context:}
            To perform a regular physical exam on the T.Rex.

        \item[]\textbf{Preconditions:}
            The T.Rex has been successfully sedated, the veterinarian is completely prepared, 
            the CGC is not in emergency mode, and all components are functioning properly.

        \item[]\textbf{Trigger:}
            The time for a physical has arrived.

        \item[]\textbf{Scenario:}
            \begin{enumerate}
                \item The CGC Station Operator dispatches the veterinarian in a self driving car
                to the edge of the enclosure closest to the current location of the T.Rex.
                \item The veterinarian requests an all-clear confirmation from the operator.
                \item The CGC Station Operator confirms sedated state of the T.Rex.
                \item The operator disengages the electricity of the panel to provide access.
                \item The veterinarian enters and travels toward the animal.
                \item The operator starts a timer.
                \item The veterinarian arrives at the location of the animal.
                \item The operator stops the timer. 
                \item The veterinarian performs a physical exam while the operator provided updates
                on the sedative state of the T.Rex.
                \item The operator alerts the veterinarian when the previously recorded elapsed time
                is approaching the approximated amount of time until the T.Rex wakes up.
                \item The veterinarian concludes the exam.
                \item The veterinarian replenishes the sedative reservoir in the T.Rex Monitor.
                \item The veterinarian travels toward the point of entry.
                \item The veterinarian exits the enclosure.
                \item The Operator confirms successful exit.
                \item The Operator reengages the electricity of the panel.                 
            \end{enumerate}

        \item[]\textbf{Exceptions:}
            \begin{itemize}
                \item[] The T.Rex is found to be in poor health.
                \item[] The sedative lasts less time than expected.
            \end{itemize}

        \item[]\textbf{Priority:}
            Essential, must be implemented.

        \item[]\textbf{When Available:}
            On Demand.

        \item[]\textbf{Frequency of Use:}
            As little as once a year.

        \item[]\textbf{Channel to Primary Actor:}
            \begin{itemize}
                \item[] Enclosure Panel, T.Rex Monitor
            \end{itemize}

        \item[]\textbf{Secondary Actors:}
            CGC Station Operator, T.Rex, Car
        
        \item[]\textbf{Channels to Secondary Actors:}
            \begin{itemize}
                \item[] CGC Station Operator: Car Intercom, Camera Network
                \item[] T.Rex: Enclosure Panel, T.Rex Monitor
            \end{itemize}

        \item[]\textbf{Open Issues:}
            \begin{itemize}
                \item[] Should the panel remain inactive while 
                the veterinarian is inside?
                \item[] Should the veterinarian simply wear an electric 
                safety suit to avoid disengagement all together?
            \end{itemize}
    \end{itemize}
    
    % another use case
    
    % another use case
    
    % ...
    
    \subsection{Zookeeper} \label{zook}
    \textit{A zookeeper may interact with the CGC in a variety of ways, but some of the
    major roles of such an actor (as with any zookeeper) are to prepare the diet of the T-Rex, 
    feed the T.Rex, to observe its behavior, or groom it.}
    \par\noindent\rule{\textwidth}{0.4pt}    
    \begin{itemize}
        \item[]\textbf{Use Case:} 
            FeedTRex                                

        \item[]\textbf{Primary Actor:}
            Zookeeper

        \item[]\textbf{Goal in Context:}
            To safely provide food for the T.Rex, whether it be live, frozen, thawed, or prepared
            prey.

        \item[]\textbf{Preconditions:}
            The CGC is not in emergency mode, and all components are fully functional.

        \item[]\textbf{Trigger:}
            It is time to feed the T.Rex.

        \item[]\textbf{Scenario:}
            \begin{enumerate}
                \item The CGC Station Operator dispatches the zookeeper in a self driving car
                to the edge of the enclosure furthest from the current location of the T.Rex.
                \item The Zookeeper requests an all-clear confirmation from the operator.
                \item The operator disengages the electricity of the panel to provide access.
                \item The Zookeeper enters and travels a significant distance into the enclosure.
                \item The Zookeeper drops off the food.
                \item The Zookeeper travels back the point of entry.
                \item The Zookeeper exits the enclosure.
                \item The Operator confirms successful exit.
                \item The Operator reengages the electricity of the panel. 
            \end{enumerate}

        \item[]\textbf{Exceptions:}
            \begin{itemize}
                \item[] There is a shortage of food on the island.
                \item[] The T.Rex is sick or injured and does not want to eat.
                \item[] The T.Rex reaches the zookeeper before the zookeeper exits the enclosure.
            \end{itemize}

        \item[]\textbf{Priority:}
            Essential, must be implemented

        \item[]\textbf{When Available:}
            On demand and via operator-zookeeper protocol

        \item[]\textbf{Frequency of Use:}
            Periodically (it can be daily, weekly, or monthly for example)

        \item[]\textbf{Channel to Primary Actor:}
            \begin{itemize}
                \item[] Enclosure Panel
            \end{itemize}

        \item[]\textbf{Secondary Actors:}
            CGC Station Operator, T.Rex, Car
        
        \item[]\textbf{Channels to Secondary Actors:}
            \begin{itemize}
                \item[] CGC Station Operator: Car Intercom, Camera Network
                \item[] T.Rex: Enclosure Panel
            \end{itemize}

        \item[]\textbf{Open Issues:}
            \begin{itemize}
                \item[] Should the panel remain inactive while 
                the zookeeper is inside?
                \item[] Should the zookeeper simply wear an electric 
                safety suit to avoid disengagement all together?
            \end{itemize}
    \end{itemize}
    
    % another use case
    
    % another use case
    
    % ...
    

% secondary actors (note that primary actors above may be secondary in some contexts)
% any devices not mentioned above
% any humans not mentioned above



\section{Definition of Terms} \label{defs} %Anas
	\paragraph{} \textit{The following is a list of definitions \footnote{Definition of 
	Terms by Anas Gauba} of the most commonly used technical terms within this 
	document, whose meaning may not be immediately apparent to the lay reader. Most 
	definitions come from no specific source; instead they are defined by the authors 
	in the context of their use in this document and originate from the vocabulary 
	shared across the general references cited \nocite{*}. In the event that a 
	definition was taken directly from a source, it is followed by a citation}
	
	\begin{list}{}{}
	    \item \textbf{CGC:} Acronym for Cretaceous Gardens Controller 
	    \item \textbf{DVR:} Acronym for Digital Video Recorder
	    \item \textbf{Electrical Conduction:} The movement of electrically charged     
	        particles through a transmission medium.
	    \item \textbf{GPS:} Global Positioning System 
	    \item \textbf{Hardwired Ethernet:} This references the latest IEEE standard for 
	        Ethernet utilizing physical cables.
	    \item \textbf{Network:} All nodes with which the CGC interacts, the links that 
	        connect them to each other and to the
	        CGC, the CGC itself, and all related databases.
	    \item \textbf{Node:} The generic term that refers to any device connected to 
	        the CGC in any way. This includes 
	        autonomous vehicles, tokens, the T.Rex monitor, all electric fence panels, 
	        all kiosks, and all cameras.
	    \item \textbf{Safely Inactive:} A state in which a vehicle is fully functional 
	        and ready to be dispatched.
	    \item \textbf{Safely Occupied:} A state in which a vehicle contains at least 
	        one person, is locked, and is ready to depart.
	    \item \textbf{Token:} An interactive device used by the visitor that grants 
	        access to locations.
	\end{list}
\pagebreak
\bibliography{../../ReferenceMaterial/BibTeX/references} % Zeke
% run latex, then bibtex, then quickbuild (on the tex file)
\end{document}
