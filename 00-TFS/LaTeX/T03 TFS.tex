\documentclass[12pt]{article}
\usepackage{xcolor}
\usepackage{sectsty}
\sectionfont{\color{DarkBlue}} 
\subsectionfont{\color{LightBlue}}
\begin{document}
\definecolor{DarkBlue}{HTML}{4a5a8a}
\definecolor{LightBlue}{HTML}{4f81bf}
\begin{titlepage}
\begin{flushleft}
\vspace*{1cm}
\Huge
\textbf{Cretaceous Gardens Controller}\\
\vspace{1cm}
\Huge
\textit{Technical Feasibility Study}\\
\vspace{1cm}
\Large
\textit{TFS Version 1.0}\\
\vspace{5cm}
\LARGE
Team \#3\\
15 October 2019
\vfill
\Huge
\textbf{CS 460 Software Engineering}
\end{flushleft}
\end{titlepage}
\normalsize
\tableofcontents
\newpage
%\section{Executive Summary}
%\paragraph{}

\section{Introduction}
\paragraph{} \textit{With new technological advancements come new opportunities and new risks. It is important for
 any entrepreneur to be able to conservatively calculate potential benefits, and to liberally
 estimate potential liabilities. With that in mind, Cretaceous Gardens (CG) ought to expect a system
 that delivers efficient service while minimizing the accumulation of loss and liability. It 
 is understood that CG has in its hands the torch of bioengineering achievement, therefore the
 feasibility of a controller (CGC) necessarily impacts the position of CG at the top of the field.
 No great feat comes without risk, therefore an ideal controller shall prioritize the safety of
 visitors to the island. Only after such a guarantee can be established, should the system then
 prioritize efficiency and profitability. Without an impeccably safe system, a long-term influx 
 of customers is unlikely and without that, there are simply no worthwhile profits to be collected.}
 
\paragraph{} The island is understood to be overlaid with a vast network of agents and equipment to
ensure the reliable connectedness of the resort. All nodes (whether it be the autonomous vehicles that are
to transport precious visitors to and from the main attraction, the electric fence enclosure that will 
house the mighty \textit{Tyrannosaurus Rex}, the system of security cameras, the reliable maintenance crew, 
or the kiosks that are to dispense every single accounted-for token) will ultimately be under the wings
of the ideal controller. Considerations should be made early with regard to broad organization of the system
(i.e. centralized, with plenty of redundancy or a distributed in order to avoid a single point of failure?).
Below is an outline of such considerations, some given by Cretaceous Gardens, and many others provided by
the developers of the system.


\section{Logistic Implications}
\paragraph{}
	\subsection{Given Constraints}
	\paragraph{} Constraints provided by CG are presented as a foundation on which to build. It should be 
	noted that it is not necessarily complete, as more may be provided in the near future. It is as likely 
	that some may be deemed unnecessary if another constraint is found to subsume them. Questions pertaining 
	to a given constraint are shown parenthetically.
	\begin{itemize}
		\item Visitors will arrive by port on the \textit{south side} of the island.
		\\(At what rate?)
		\item The main attraction will be located on the north side of the island. 
		\\(Are there multiple routes?)
		\item The main attraction will be a \textit{Tyrannosaurus Rex} within a massive electric fence enclosure. 
		\\(Can it be monitored on a per-panel basis?)
		\item An automatic payment kiosk will provide a unique token \textit{or} ticket to each visitor. 
		\\(What if the item is damaged or lost?)
		\item The token \textit{or} ticket will grant vehicle access \textit{and} exhibit viewing access.
		\item Self-driving car(s) are to stop at the \textit{south end} in order to drive \textit{at most} 10 people. 
		\\(How many cars at a time?)
		\item All cars are to lock securely \textit{before} transporting visitors to the north-end.
		\item All cars are to unlock \textit{after} arriving at the exhibit.
		\item All cars are to alert visitors when their allotted time to be at the exhibit has run out. 
		\\(Will this alert be different than an emergency alert?)
		\\(What should happen if a visitor ignores the "time-up" alert?)
		\\(Who or what will enforce this?)
		\item The vehicle is not to leave until all visitors are inside the vehicle. 
		\\(Even during an emergency?)
		\item The positions of all cars are to be accounted for at all times. 
		\\(To what precision?)
		\item There will be \textit{at least} one video camera. 
		\\(Will it be closed circuit?) 
		\\(What kind of camera?)
		\item Doors are to lock securely with all guests inside \textbf{before} returning the \textit{south end}.
		\item There is to be a \textit{very strict} protocol in the event of an enclosure failure.
		\item The emergency alarm is to sound at both ends of the island.
		\item The system is to be responsible for the quick transport of all visitors to the south end in the event of such an emergency.
		\\(What should be done if casualties occur?)
		\\(Should an optional manual driving mode be made available to willing visitors in such an emergency?)
		\\(What should happen in the event of path obstructions?)
		\\(How resilient will the vehicles be in the local terrain?) 
	\end{itemize}
	\subsection{Additional Constraints}
	\paragraph{} there is assumed to be employees
	\begin{itemize}
		\item[]
		\item[]
		\item[]
		\item[]
		\item[]
		\item[]
		\item[]
		\item[]
		\item[]
		\item[]
		\item[]
		\item[]
		\item[]
		\item[]
		\item[]
		\item[]
		\item[] 
	\end{itemize}
\section{Findings}
\paragraph{}
	\subsection{General}
	\paragraph{} 
	- We will need a working camera network. all cameras should be viewable from control system
	- the self payrobusting kiosk will present tokens(tickets) that are unique 
            - form must be filled out with name, emergency contact, and waiver
            - accept credit card
            - accept cash
            - upon acceptance the token with unique ID will be tied to 
            must be able to configure the token and revuild RFID information
        - (my opinion) I believe that these tokens should be devices that have:
            - map (digital?) used to help guide
            - ***NEEDED*** GPS - will be tracked
            - ***NEEDED*** reprogrammable RFID ship
            - Speaker - to help instruct 
        - Self-Driving car (contracted out) (how many total cars do we have?)
            - must work with out interface and logic for use in normal mode
            - must work with out interface and logic for use in Emergency mode
            - must include GPS
            - must include loud Speaker
            - must include a unique car ID
            - Must include an RFID reader
            - there must be a call button to talk with the Control Station
            - door sensors

	\paragraph{} 
	\paragraph{}
	
	\subsection{Safety}
	\paragraph{}
	        - Electric Fence status needs to be available.
        - When Electric fence fails emergency mode MUST be activated
        - Cars will have a safety features
            - they will lock securely when driving,
            - they must know what visitors they drive over to T-Rex pit (using the tokens)
            - they sound an alarm and send notifications to tokens that they are keeping track of after alloted time to bring back to south-end
            - They will ONLY leave after all visitors that it drove over are inside the car
            - there will be X amount of dedicated patrol cars used by employees that go back and forth from south to north end
	\paragraph{}

	\subsection{Emergency}
	\paragraph{}
	
	\subsection{Maintenance}
	\paragraph{}
	- Emergency Mode triggered from the electric fence outage or from testing in maintanance mode
        - Alarms triggered everywhere!
        - There will be X amount of self driving cars that will always be located at the north-side only used for emergency mode
        - Cars will loosen the requirements and will begin accepting  any passengers and will begin leaving every X minutes whether full or not
        - all cars that are NOT currently in manual override or driving passengers back to the south end will be dispatched to the north-end
        - after emergency mode is initiated - even if the fence comes back up, emergency mode should not exit until manually triggered by a human employee
    - Maintanance
        - Cars 
            - will need to alert the Control station when they need maintanance if it can drive they should return to the south lot and taken out of production
            - any passengers that were on this vehicle can board a new vehicle
            - if a car breaks down on the island a maintanance crew will ride a new self-driving vehicle to the broken down one and help move visiters into new vehicle.
            
        - Fence
            - the power to the fence should have extra redundencies which allows for seemless maintanance with absolutely NO power outage
        - camera systems
            - camera should be hot swapable and uplinks should be redundent
        - tokens 
            - these may need to be maintaned and upgraded and replaced over time.
        - kiosks
            - these should notify the control station of need for maintanance or have a manual override to turn on maintanance mode
	\paragraph{}
	
\section{Conclusions}
\paragraph{}
\paragraph{}
\paragraph{}

\bibliography{../ReferenceMaterial/references}
\end{document}