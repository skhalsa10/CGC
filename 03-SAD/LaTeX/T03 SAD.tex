\documentclass[12pt]{article}
\usepackage[english]{babel}
\usepackage[numbers]{natbib}
\usepackage{graphicx}
\usepackage{sectsty}
\usepackage{float}
\usepackage[table]{xcolor}
\usepackage{tabularx}
\usepackage{arydshln}

\bibliographystyle{apalike}
\setcitestyle{open={[},close={]}}
\sectionfont{\color{DarkBlue}} 
\subsectionfont{\color{LightBlue}}
\subsubsectionfont{\color{LightBlue}}
\paragraphfont{\color{LightBlue}}
\subparagraphfont{\color{LightBlue}}
\begin{document}
\definecolor{DarkBlue}{HTML}{4a5a8a} 
\definecolor{LightBlue}{HTML}{4f81bf}
\begin{titlepage}
    \begin{flushleft}
        \vspace{1cm} \Huge  \textbf{Elevator Control System}\\
        \vspace{1cm} \Huge  \textit{Software Architecture Design}\\
        \vspace{1cm} \Large \textit{SAD Version 2.0}\\
        \vspace{5cm} \LARGE         Team \#3\\ 
                                    8 October 2019
        \vfill       \Huge  \textbf{CS 460 Software Engineering}
    \end{flushleft}
\end{titlepage}
\normalsize 
\tableofcontents
\pagebreak

\section{Introduction} \label{intro}
\paragraph{} Good software is identified by the end user for its features and functionality, by the client
for its profitability and maintainability, and by the programmer for its legibility and clarity. It should be clear
that all three pillars ultimately characterize good software, but more importantly that the three are interdependent.
The developer must then guarantee all of the above for the sake of all entities involved. A top down approach has been
taken up to this point. The feasibility study asked answered the fundamental question: Can it be done? The requirements
definition document was then passed the baton and answered the next logical question: Within what parameters can it be done?
The software specification document then answered: What is the desired behavior of the system? Now we may answer: 
How will ensure the desired behavior?

\paragraph{} The ultimate goal is to ensure an efficient, safe, and maintainable implementation of the Cretaceous Gardens Controller (CGC) software\footnote{Introduction by Zeke.}. To that end, all objects are illustrated in their proper contexts, and their crucial functions have been delineated as clearly as possible while simultaneously allowing the programmer enough flexibility so as to not stifle his or her creative process. 

\paragraph{} This is a road map for the eventual implementation of the system. It details the most relevant objects and their relationships in the form of diagrams. Explanations accompany all diagrams for the sake of clarity.

\section{Design Overview} \label{over}
\paragraph{} \textit{The class architecture presented here \footnote{Diagrams by 
Siri, Anas, Santi, and Zeke.} aims to maximize efficiency, maintainability, and 
safety. Without an efficient system, its safety may be compromised due to unnecessary 
delays between components. Maintainability can impact a safe implementation of the 
system if the system is permeable to programmer errors. The decoupling of concerns and 
a solid hierarchy are paramount. The design has been color-coded to increase readability. 
The colors are only for the sake of distinguishing one component from another. The component 
specification diagrams in the following section inherit these colors. Small red arrows point 
to objects that may be triggered by an event. Said objects are virtual devices and iconifications 
of their physical triggers have been connected to them with bidirectional blue arrows.}

%\begin{figure}[H]
%    \centerline{\includegraphics[scale=.75]{EntireSystem.png}}
%    \caption{An overview of the Cretaceous Gardens Control System. Red arrows indicate that an object may be triggered by an event. The
%    event is represented by the icon connected to the object.}
%    \label{fig:EntireSystem}
%\end{figure}    
%      
%\begin{figure}[H]
%    \centerline{\includegraphics[scale=.70]{CGC.png}}
%    \caption{The Cretaceous Gardens Control System}
%    \label{fig:cgc}
%\end{figure}
 
%\begin{figure}[H]
%    \centerline{\includegraphics[scale=.70]{someDiagram.png}}
%    \caption{Caption}
%    \label{fig:somelabel}
%\end{figure}
%
%\begin{figure}[H]
%    \centerline{\includegraphics[scale=.70]{someDiagram.png}}
%    \caption{Caption}
%    \label{fig:somelabel}
%\end{figure}
%
% .
% .
% .

\section{Component Specifications}
\paragraph{} \textit{Here are the class specifications \footnote{Component specifications by Anas and Siri} for objects 
found in Section \ref{over}. The most important attributes and functions are given and, where appropriate, any expected 
preconditions, parameters, return values, and post conditions are included. Explanations are given for each component, but 
it should be noted that helper functions are not shown, as they are entrusted with the programmer. Important fields and helper
fields are treated analogously.}

\begin{table}[H]
\begin{tabularx}{\hsize}{|Y|Y|}
    \hline
    \rowcolor{nicegreen}
    \multicolumn{2}{|c|}{\textbf{Some Class Name}} \\ % Title of Class
    \hline
    \hline
    \multicolumn{2}{|c|}{\textbf{Attributes}}      \\
    \hline
    \textbf{attribute 1} & brief description \\
    \textbf{attribute 2} & brief description \\
    \textbf{.} & brief description \\
    \textbf{.} & brief description \\
    \textbf{.} & brief description\\
    \multicolumn{2}{|c|}{\textbf{Functions}} \\
    \hline
    \textbf{function1()} & $\rightarrow$ brief description \\
    \textbf{function2()} & $\rightarrow$ brief description \\
    \textbf{.} & $\rightarrow$ brief description \\
    \textbf{.} & $\rightarrow$ brief description \\
    \textbf{.} & $\rightarrow$ brief description\\
\end{tabularx}
\end{table}
\par\noindent\rule{\textwidth}{0.4pt}

\section{Sample Use Case} \label{samp}
\paragraph{} \textit{A broad overview of use cases begins this section and it is followed 
by detailed case descriptions. Human actors are denoted by small stick figures and have the 
same color scheme as the box that contains their labels. The section later features use case 
diagrams for each actor and a detailed sample of use cases.}
%\begin{figure}[H]
%    \centerline{\includegraphics[scale=.30]{ECSUseCaseUML.png}}
%    \caption{Shows the a general view of use cases as a diagram. The primary actors are color coded on the left, and exclusive
%    goals share the same color scheme. Secondary actors are shown on the right. Goals that may be had by various actors have a 
%    grey color scheme. The dashed lines indicate inclusions.}
%    \label{fig:usecasediagram}
%\end{figure}


\section{Design Constraints} \label{cons}
\paragraph{} \textit{Due to the real-time nature of the system, there exist some additional 
constraints\footnote{Design Constraints by Anas.}. Namely, it must be the case that all data 
structures concerning the safety controls are as fast as possible but also that they are capable 
of prioritizing all signals in the best way possible.}

    \subsection{Safety}
    \paragraph{} The safety is highly prioritized in our design of the ECS. We have 
    considered associating the fire alarm directly with the ECS because in the case 
    of the fire alarm triggering, this event has a very high priority and the ECS 
    should immediately react to this event. When it comes to elevator and bay safety, 
    each elevator and bay will have set of Safety Controls that they can communicate 
    through and the ECS monitors the situation from the top. By ensuring safety for 
    both bays and the elevators, the ECS operation will be carried smoothly without 
    hurting the passengers.

    \subsection{Implementation Guidelines}
    \paragraph{} According to the design, we suggest programmers to use some sort of 
    Concurrent safe Messaging Queue for the communication between sensing objects and 
    their associated parent objects. We also recommend using concurrent safe Priority 
    Queue for the ECS, so the ECS can react based on the certain given priority. The 
    priority should be considered because in the case of an emergency, the priority 
    for that event should be at the very top so that the ECS should immediately react 
    to it by closing its doors and going to first floor if its not there already.

\section{Definition of Terms} \label{defs}
\paragraph{} \textit{The following is a list of definitions contain the most commonly used 
technical terms within this document, whose meaning may not be immediately apparent to the 
lay reader. Most definitions are defined by the authors for use within the context of this 
document. Some may originate from vocabulary shared across the general references cited \nocite{*}. 
In the event that a definition was taken directly from a source, it is followed by a citation.
\footnote{This list is mostly a reduction of the term list found in the preceding Software 
Design Specification document.}}

\begin{list}{}{}
    \item \textbf{CGC:} Acronym for Cretaceous Gardens Controller 
    \item \textbf{DVR:} Acronym for Digital Video Recorder
    \item \textbf{Electrical Conduction:} The movement of electrically charged particles through a transmission medium.
    \item \textbf{GPS:} Global Positioning System 
    \item \textbf{Hardwired Ethernet:} This references the latest IEEE standard for Ethernet utilizing physical cables.
    \item \textbf{Network:} All nodes with which the CGC interacts, the links that connect them to each other and to the CGC, the CGC itself, and all related databases.
    \item \textbf{Node:} The generic term that refers to any device connected to the CGC in any way. This includes autonomous vehicles, tokens, the T.Rex monitor, all electric fence panels, all kiosks, and all cameras.
    \item \textbf{Safely Inactive:} A state in which a vehicle is fully functional and ready to be dispatched.
    \item \textbf{Safely Occupied:} A state in which a vehicle contains at least one person, is locked, and is ready to depart.
    \item \textbf{Token:} An interactive device used by the visitor that grants access to locations.
\end{list}
\pagebreak
\bibliography{../../ReferenceMaterial/BibTeX/references}
% run latex, then bibtex, then quickbuild (on the tex file)
\end{document}
