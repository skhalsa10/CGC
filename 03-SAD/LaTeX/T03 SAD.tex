\documentclass[12pt]{article}
\usepackage[english]{babel}
\usepackage[numbers]{natbib}
\usepackage{graphicx}
\usepackage{xcolor}
\usepackage{sectsty}
\usepackage{float}
\bibliographystyle{apalike}
\setcitestyle{open={[},close={]}}
\sectionfont{\color{DarkBlue}} 
\subsectionfont{\color{LightBlue}}
\subsubsectionfont{\color{LightBlue}}
\paragraphfont{\color{LightBlue}}
\subparagraphfont{\color{LightBlue}}
\begin{document}
\definecolor{DarkBlue}{HTML}{4a5a8a} 
\definecolor{LightBlue}{HTML}{4f81bf}
\begin{titlepage}
	\begin{flushleft}
		\vspace{1cm} \Huge  \textbf{Elevator Control System}\\
		\vspace{1cm} \Huge  \textit{Software Architecture Design}\\
		\vspace{1cm} \Large \textit{SAD Version 2.0}\\
		\vspace{5cm} \LARGE         Team \#3\\ 
		                            8 October 2019
		\vfill       \Huge  \textbf{CS 460 Software Engineering}
	\end{flushleft}
\end{titlepage}
\normalsize 
\tableofcontents
\pagebreak

\section{Introduction} \label{intro}
\paragraph{} Good software is identified by the end user for its features and functionality, by the client
for its profitability and maintainability, and by the programmer for its legibility and clarity. It should be clear
that all three pillars ultimately characterize good software, but more importantly that the three are interdependent.
The developer must then guarantee all of the above for the sake of all entities involved. A top down approach has been
taken up to this point. The feasibility study asked answered the fundamental question: Can it be done? The requirements
definition document was then passed the baton and answered the next logical question: Within what parameters can it be done?
The software specification document then answered: What is the desired behavior of the system? Now we may answer: 
How will ensure the desired behavior?

\paragraph{} The ultimate goal is to ensure an efficient, safe, and maintainable implementation of the 
Elevator Control System software\footnote{ Introduction by Zeke.}. To that end, all objects are illustrated in their proper contexts, and
their crucial functions have been been delineated as clearly as possible while simultaneously allowing 
the programmer enough flexibility so as to not stifle his or her creative process. 

\paragraph{} This is a road map for the eventual implementation of the system. It details the most relevant objects 
and their relationships in the form of diagrams. Explanations accompany all diagrams for the sake of clarity.

\section{Design Overview} \label{over} %design approach, design diagram with interfaces, diagram explanation
	\paragraph{} \textit{The class architecture presented here \footnote{ Diagrams by Siri, Anas, Santi, and Zeke.}
	aims t maximize efficiency, maintainability, and safety. Without an efficient system, its safety may be compromised 
	due to unnecessary delays between components. Maintainability can impact a safe implementation of the system 
	if the system is permeable to programmer errors. The decoupling of concerns and a solid hierarchy are paramount. 
    The design has been color-coded to increase readability. The colors are only for the sake of distinguishing one 
	component from another. The component specification diagrams in the following section inherit these colors. Small 
	red arrows point to objects that may be triggered by an event. Said objects are virtual devices and iconifications
	of their physical triggers have been connected to them with bidirectional blue arrows.} 

	\paragraph{} Figure \ref{fig:EntireSystem} shows the entire system. The root object is the ECS component, 
	communicates with all its children. Connected hardware and interfaces are are shown as well and are primarily leaf nodes.
	The rest of the figures are snapshots at various levels of the tree.

    \begin{figure}[H]
  		\centerline{\includegraphics[scale=.75]{EntireSystem.png}}
  		\caption{An overview of the Elevator Control System. Red arrows indicate that an object may be triggered by an event. The
  		event is represented by the icon connected to the object.}
  		\label{fig:EntireSystem}
	\end{figure}	
	
	
		\paragraph{} Figure \ref{fig:ecs} shows the root node of the whole system, the ECS component. Figures \ref{fig:bay} 
		through \ref{fig:elevator} show the next level down, and the rest delve deeper into the structure. More detailed descriptions
		of each the diagrams can be found in their captions.
		\begin{figure}[H]
  			\centerline{\includegraphics[scale=.70]{ECS.jpeg}}
  			\caption{The Elevator Control System}
  			\label{fig:ecs}
		\end{figure}
	
		\paragraph{} Figure \ref{fig:bay} shows the \textit{Bay} virtual device and its child virtual devices, the \textit{Bay Button Panel}, 
		a \textit{Door Sensor}. It also possesses a \textit{Safety Control} which is the software component that ensures the safe function
		of the virtual devices. Figure \ref{fig:safetycontrol} shows a more detailed view of the \textit{Safety Control} for the bay.	
		\begin{figure}[H]
			\centerline{\includegraphics[scale=.70]{Bay.jpeg}}
			\caption{The Bay}
			\label{fig:bay}
		\end{figure}
			\begin{figure}[H]
				\centerline{\includegraphics[scale=.70]{SafetyControl.jpeg}}
				\caption{The Safety Control}
				\label{fig:safetycontrol}
			\end{figure}
	
		\paragraph{} Figure \ref{fig:elevator} shows the \textit{Elevator} component, which has many parts. The \textit{Key Panel} shown 
		in Figure \ref{fig:keypanel} provides an interface for handling the use of emergency and executive keys. The \textit{Sound System} 
		in Figure \ref{fig:soundsystem} controls the audio and intercom functionalities. The final components are the \textit{Cabin Button Panel} 
		(Figure \ref{fig:cabinbuttonpanel}) and the \textit{Cabin Door} (Figure \ref{fig:cabindoor}). These last two are virtual devices.
		\begin{figure}[H]
			\centerline{\includegraphics[scale=.70]{Elevator.jpeg}}
			\caption{The Elevator}
			\label{fig:elevator}
		\end{figure}
			\begin{figure}[H]
				\centerline{\includegraphics[scale=.70]{KeyPanel.jpeg}}
				\caption{The Key Panel}
				\label{fig:keypanel}
			\end{figure}	
			\begin{figure}[H]
				\centerline{\includegraphics[scale=.65]{CabinSoundSystem.jpeg}}
				\caption{The Cabin Sound System}
				\label{fig:soundsystem}
			\end{figure}
			\begin{figure}[H]
				\centerline{\includegraphics[scale=.65]{CabinButtonPanel.jpeg}}
				\caption{The Cabin Button Panel}
				\label{fig:cabinbuttonpanel}
			\end{figure}
			\begin{figure}[H]
				\centerline{\includegraphics[scale=.70]{CabinDoor.jpeg}}
				\caption{The Cabin Door}
				\label{fig:cabindoor}
			\end{figure}


\section{Component Specifications}
\paragraph{} \textit{Here are the class specifications \footnote{Component specifications by Anas and Siri} for objects 
found in Section \ref{over}. The most important attributes and functions are given and, where appropriate, any expected 
preconditions, parameters, return values, and post conditions are included. Explanations are given for each component, but 
it should be noted that helper functions are not shown, as they are entrusted with the programmer. Important fields and helper
fields are treated analogously.}


	\subsection*{ECS Class}
	\paragraph{} The ECS class controls all other classes in the system. All sub-component states are known to the ECS and it 
	possesses the primary algorithms that coordinate all sub-components.
	\begin{table}[H]
		\begin{tabular}{lp{12cm}}
			\textbf{Attributes} & \\ 
			Elevators [0 .. 3]  & Instances of virtual devices that represent elevators \\
		   	Bay[1 .. 20]        & Instances of virtual devices that represent bays \\
			currentMode         & Indicates the mode of the system $\{$alarmMode, emergencyMode, custodialMode, maintenanceMode, normalMode$\}$ \\
		\end{tabular}
	\end{table}     
	\begin{table}[H]
		\begin{tabular}{lp{12cm}}
			\textbf{Functions}                  & \\
			selectElevator()                    & $\rightarrow$ selects best elevator upon request. \\
			sendElevator(Elevator)        		& $\rightarrow$ sends the elevator to the target bay.\\
			checkSafetyControl(Bay, BayNumber) 	& $\rightarrow$ performs a safety check on a bay.\\
			checkSafetyControl(Elevator) 		& $\rightarrow$ performs a safety check on a cabin.\\
			changeMode() 						& $\rightarrow$ toggles between normal mode and emergency mode.\\
			openDoor() 							& $\rightarrow$ opens cabin door and requests the same of bay analog\\
			closeDoor() 						& $\rightarrow$ closes cabin door and requests the same of bay analog\\
			activateEmergencyMode() 			& $\rightarrow$ explicitly enters emergency mode.\\
			initialize() 						& $\rightarrow$ initializes the system.\\
			shutdown() 							& $\rightarrow$ shuts down the system.\\
			reset() 							& $\rightarrow$ resets the system.\\
		\end{tabular}
	\end{table}
	\begin{figure}[H]
  		\centerline{\includegraphics[scale=.70]{ECS_class.jpeg}}
  		\caption{The Elevator Control System Class}
  		\label{fig:ecs_class}
	\end{figure}
	\par\noindent\rule{\textwidth}{0.4pt}
	
	\subsection*{Alarm Class}
    \begin{figure}[H]
  		\centerline{\includegraphics[scale=1]{FireAlarm_class.jpeg}}
  		\caption{The Fire Alarm Class}
  		\label{fig:FireAlarm_class}
	\end{figure}
	\par\noindent\rule{\textwidth}{0.4pt}
	
	\subsection*{Bay Class}
	\paragraph{}The Bay class is a virtual device that mitigates request flow between the physical bay and the ECS.
	\begin{table}[H]
        \begin{tabular}{lp{12cm}}
        	\textbf{Attributes}    & \\
        	Message System         & a data structure \\
        	bayNum                 & number of the bay \\
        	floorNum               & number of the floor \\
        	bayButtons             & instance of Bay Button Panel \\
        	Safety Control         & instance of Safety Control \\
        	Bay Doors              & Instance of Bay Doors \\
        \end{tabular}
	\end{table}
    \begin{table}[H]
    	\begin{tabular}{lp{12cm}}
    		\textbf{Functions}          & \\
        	requestElevator(Direction)  & $\rightarrow$ requests an elevator specifying target floor direction.\\
	        buttonPressed()             & $\rightarrow$ listens for button presses.\\
	        openDoors()                 & $\rightarrow$ opens bay doors.\\
	        closeDoors()                & $\rightarrow$ opens bay doors.\\
		\end{tabular}
	\end{table}
    \begin{figure}[H]
  		\centerline{\includegraphics[scale=.70]{Bay_class.jpeg}}
  		\caption{The Bay Class}
  		\label{fig:Bay_class}
	\end{figure}
    \begin{figure}[H]
  		\centerline{\includegraphics[scale=.70]{BayButtonPanel_class.jpeg}}
  		\caption{The Bay Button Panel Class}
  		\label{fig:BayButtonPanel_class}
	\end{figure}
	\begin{figure}[H]
  		\centerline{\includegraphics[scale=.70]{BayDoors_class.jpeg}}
  		\caption{The Bay Doors Class}
  		\label{fig:BayDoors_class}
	\end{figure}
	\par\noindent\rule{\textwidth}{0.4pt}
	
	\subsection*{Elevator Class}
	\paragraph{} The Elevator class encapsulates components that constitute the elevator,  
	has communicates with physical components via virtual devices, and possesses some control logic
	for the cabin.
			
	\begin{table}[H]
        \begin{tabular}{lp{12cm}}
        	\textbf{Attributes} & \\
        	Cabin Number    	& an identifier \\
		    Current Floor     	& the floor on which the cabin resides \\
		    Direction           & the direction in which the cabin is currently moving\\
		    Final Floor     	& the target floor\\
		    Cabin Buttons     	& an instance of Cabin Buttons class\\
		    Key Panel           & an instance of Key Panel class\\
		    Safety Control    	& an instance of Safety Control class\\
		    Emergency Brake 	& a virtual device to represent the elevator brake\\
		    ElevatorMoving    	& indicates whether or not the cabin is in motion \\
		    ExecutiveMode    	& indicates whether or not an executive key is detected \\
		    SoundSystem         & instance of Sound System class \\
	    \end{tabular}
	\end{table}
	\begin{table}[H] 
	    \begin{tabular}{lp{12cm}}
	    	\textbf{Functions}       & \\
			validateEmergencyKey()   & $\rightarrow$ validates emergency key\\
		    validateExecutiveKey()   & $\rightarrow$ validates executive key\\
		    isJukeBoxPlaying()       & $\rightarrow$ indicates whether or not the jukebox is playing\\
		    setSpeakerConfig(Volume) & $\rightarrow$ sets speaker volume\\
		    isIntercomActivated()    & $\rightarrow$ indicates whether or not intercom is active\\
		    changeIntercomStatus()   & $\rightarrow$ toggles intercom state\\
		    isFloorArrived()         & $\rightarrow$ indicates whether or not a request has been fulfilled\\
		    moveCabin(floor Number)  & $\rightarrow$ moves a cabin to the given floor\\
		    modifyDoors()            & $\rightarrow$ toggle door state\\
		    applyBrake()             & $\rightarrow$ activates the emergency brake\\
		    buttonPressed()          & $\rightarrow$ listens for a button press\\
		    emergencyCall()          & $\rightarrow$ sends emergency signal and starts the intercom\\
	    \end{tabular}
	\end{table}
    \begin{figure}[H]
  		\centerline{\includegraphics[scale=.70]{Elevator_class.jpeg}}
  		\caption{The Elevator Class}
  		\label{fig:Elevator_class}
	\end{figure}
    \begin{figure}[H]
  		\centerline{\includegraphics[scale=.70]{FloorDisplay_class.jpeg}}
  		\caption{The Floor Display Class}
  		\label{fig:FloorDisplay_class}
	\end{figure}
    \begin{figure}[H]
  		\centerline{\includegraphics[scale=.70]{EmergencyBrake_class.jpeg}}
  		\caption{The Emergency Brake Class}
  		\label{fig:EmergencyBrake_class}
	\end{figure}
    \begin{figure}[H]
  		\centerline{\includegraphics[scale=.70]{CabinMotor_class.jpeg}}
  		\caption{The Cabin Motor Class}
  		\label{fig:CabinMotor_class}
	\end{figure}
	\begin{figure}[H]
  		\centerline{\includegraphics[scale=.70]{CabinDoor_class.jpeg}}
  		\caption{The Cabin Door Class}
  		\label{fig:CabinDoor_class}
	\end{figure}
	\par\noindent\rule{\textwidth}{0.4pt}

	\subsection*{Cabin Sound System Class}
	\paragraph{} The Cabin Sound System class is a virtual device that handles sound from the jukebox and intercom signals. 
	\begin{table}[H]
		\begin{tabular}{lp{12cm}}
			\textbf{Attributes} & \\ 
			Speaker   & instance of virtual device that represents the speaker \\
		   	Intercom  & instance of virtual device that represents the intercom \\
			JukeBox   & instance of virtual device that represents the jukebox \\
		\end{tabular}
	\end{table}     
	\begin{table}[H]
		\begin{tabular}{lp{12cm}}
			\textbf{Functions}      & \\
			isJukeBoxPlaying()      & $\rightarrow$ checks whether or not jukebox is playing\\
			changeVolume(volume)    & $\rightarrow$ sets volume to the given value\\
			isIntercomActivated() 	& $\rightarrow$ checks whether or not the intercom is active\\
			setIntercomStatus() 	& $\rightarrow$ toggle intercom status\\
			startPlaying()			& $\rightarrow$ play music\\
		\end{tabular}
	\end{table}
    \begin{figure}[H]
  		\centerline{\includegraphics[scale=.70]{CabinSoundSystem_class.jpeg}}
  		\caption{The Cabin Sound System Class}
  		\label{fig:CabinSoundSystem_class}
	\end{figure}
    \begin{figure}[H]
  		\centerline{\includegraphics[scale=.70]{Speaker_class.jpeg}}
  		\caption{The Speaker Class}
  		\label{fig:Speaker_class}
	\end{figure}
    \begin{figure}[H]
  		\centerline{\includegraphics[scale=.70]{Intercom_class.jpeg}}
  		\caption{The Intercom Class}
  		\label{fig:Intercom_class}
	\end{figure}
    \begin{figure}[H]
  		\centerline{\includegraphics[scale=.70]{Jukebox_class.jpeg}}
  		\caption{The Jukebox Class}
  		\label{fig:Jukebox_class}
	\end{figure}
	\par\noindent\rule{\textwidth}{0.4pt}

	\subsection*{Key Panel Class}
	\paragraph{} The Key Panel class is virtual device to detect the presence of physical keys (emergency and executive).
	\begin{table}[H]
		\begin{tabular}{lp{12cm}}
			\textbf{Attributes} & \\ 
			RFID Control           & instance of virtual device that represents the RFID reader \\
		   	Key Emergency Control  & instance of Key Emergency class\\
		\end{tabular}
	\end{table}     
	\begin{table}[H]
		\begin{tabular}{lp{12cm}}
			\textbf{Functions}      & \\
			isKeyDetected()         & $\rightarrow$ checks whether or not a key has been detected\\
			getTypeKeyDetected()    & $\rightarrow$ returns the type of key detected\\
		\end{tabular}
	\end{table}
    \begin{figure}[H]
  		\centerline{\includegraphics[scale=.70]{KeyPanel_class.jpeg}}
  		\caption{The Key Panel Class}
  		\label{fig:KeyPanel_class.}
	\end{figure}
    \begin{figure}[H]
  		\centerline{\includegraphics[scale=.70]{RFIDControl_class.jpeg}}
  		\caption{The RFID Control Class}
  		\label{fig:RFIDSensor_class}
	\end{figure}
    \begin{figure}[H]
  		\centerline{\includegraphics[scale=.70]{EmergencyKey_class.jpeg}}
  		\caption{The Emergency Key Class}
  		\label{fig:EmergencyKey_class}
	\end{figure}
	\par\noindent\rule{\textwidth}{0.4pt}
	
	\subsection*{Cabin Button Panel Class}
	\paragraph{} The Cabin Button Panel class is a virtual device to link the ECS with the physical button panel within the cabin.
	\begin{table}[H]
		\begin{tabular}{lp{12cm}}
			\textbf{Attributes} & \\ 
			Alarm Button        & instance of virtual device that represents the alarm button \\
		   	Door Open Button    & instance of virtual device that represents the door open button \\
			Door Close Button   & instance of virtual device that represents the door close button \\
			Call Button         & instance of virtual device that represents the call button \\
			Floor Button        & instance of virtual device that represents a floor button \\
		\end{tabular}
	\end{table}     
	\begin{table}[H]
		\begin{tabular}{lp{12cm}}
			\textbf{Functions}      & \\
			getButtonPressed()      & $\rightarrow$ returns the button pressed\\
		\end{tabular}
	\end{table}
    \begin{figure}[H]
  		\centerline{\includegraphics[scale=.70]{CabinButtonPanel_class.jpeg}}
  		\caption{The Cabin Button Panel Class}
  		\label{fig:CabinButtonPanel_class}
	\end{figure}
    \begin{figure}[H]
  		\centerline{\includegraphics[scale=.70]{DoorMotor_class.jpeg}}
  		\caption{The Door Motor Class}
  		\label{fig:DoorMotor_class}
	\end{figure}
	\par\noindent\rule{\textwidth}{0.4pt}
	
	\subsection*{Safety Control Class Class}
	\paragraph{} The Safety Control class communicates with all safety hardware (motors, sensors, brakes, etc.)
	and holds the logic to meaningfully interpret signals from the hardware.
	\begin{table}[H]
		\begin{tabular}{lp{12cm}}
			\textbf{Attributes}     & \\ 
			Alignment Control       & controls alignment sensors via virtual devices \\
		   	Accelerometer Control   & controls accelerometers via virtual devices \\
			Weight Control          & controls weight sensors via virtual devices \\
			Speed Control           & controls speedometers sensors via virtual devices \\
			Light Curtain Control   & controls light curtains via virtual devices \\
			Alignment Control   &
		\end{tabular}
	\end{table}     
	\begin{table}[H]
		\begin{tabular}{lp{12cm}}
			\textbf{Functions}             & \\
			checkElevatorSafetyControls()  & $\rightarrow$ performs a system wide device check\\
		\end{tabular}
	\end{table}
    \begin{figure}[H]
  		\centerline{\includegraphics[scale=.70]{SafetyControl_class.jpeg}}
  		\caption{The Safety Control Class}
  		\label{fig:SafetyControl_class}
	\end{figure}
    \begin{figure}[H]
  		\centerline{\includegraphics[scale=.70]{SpeedControl_class.jpeg}}
  		\caption{The Speed Sensor Class}
  		\label{fig:speedSensor_class}
	\end{figure}
    \begin{figure}[H]
  		\centerline{\includegraphics[scale=.70]{LightCurtainControl_class.jpeg}}
  		\caption{The Light Curtain Class}
  		\label{fig:LightCurtainSensor_class}
	\end{figure}
    \begin{figure}[H]
  		\centerline{\includegraphics[scale=.70]{AlignmentControl_class.jpeg}}
  		\caption{The Alignment Sensor Class}
  		\label{fig:AlignmentSensor_class}
	\end{figure}
    \begin{figure}[H]
  		\centerline{\includegraphics[scale=.70]{WeightControl_class.jpeg}}
  		\caption{The Weight Sensor Class}
  		\label{fig:WeightSensor_class}
	\end{figure}
    \begin{figure}[H]
  		\centerline{\includegraphics[scale=.70]{AccelerationControl_class.jpeg}}
  		\caption{The Weight Sensor Class}
  		\label{fig:Accelerometer_class}
	\end{figure}
\pagebreak	

\section{Sample Use Case} \label{samp}
\paragraph{} \textit{A broad overview of use cases begins this section and it is followed by detailed case descriptions. Human actors
are denoted by small stick figures and have the same color scheme as the box that contains their labels. Guests of the hotel are colored 
green. An executive guest (denoted by a darker border than the one for the non-executive guest) subsumes the non-executive guest role.}
\begin{figure}[H]
	\centerline{\includegraphics[scale=.30]{ECSUseCaseUML.png}}
	\caption{Shows the a general view of use cases as a diagram. The primary actors are color coded on the left, and exclusive
	goals share the same color scheme. Secondary actors are shown on the right. Goals that may be had by various actors have a 
	grey color scheme. The dashed lines indicate inclusions.}
	\label{fig:usecasediagram}
\end{figure}
	\pagebreak
	\subsection*{Typical Scenarios} This section contains textual descriptions of the most common use cases for the ECS in detail.
	Travel from one floor to another is described for executive guests, non-executive guests. Guest communication with the front desk
	is also given, as are a few custodial scenarios.
		\par\noindent\rule{\textwidth}{0.4pt}
		\begin{itemize} % go to another floor
			\item[] \textbf{Use Case:} \textit{GoToAnotherFloor}
			\item[] \textbf{Primary actor:} Non-Executive Guest (NEG)
			\item[] \textbf{Goal in context:} To travel from the current floor to another, non-executive, floor.
			\item[] \textbf{Preconditions:} System is not in maintenance, nor emergency mode, nor alarm mode. 
			Elevators, bays, and all mechanical parts can be in any safe state (aligned, below weight capacity, 
			no obstructions, etc.).
			\item[] \textbf{Trigger:} The non-executive guest decides to travel to another floor.
			\item[] { \textbf{Scenario:}
		        \begin{enumerate}
		        	\item NEG observes the elevator bays.
		        	\item NEG presses one or more buttons to request an elevator.
		        	\item NEG enters first elevator to open its doors.
		        	\item NEG presses one or more buttons to indicate the target floor (one or more 
		        	button presses may have been erroneous)
		        	\item{NEG sees someone running toward the elevator
		        		\begin{enumerate}
		        			\item NEG presses the "open doors" button
		        			\item NEG presses the  "close doors" button
		        		\end{enumerate}}
		        	\item NEG travels to target floor with zero or more stops in between.
		        	\item NEG exits the elevator upon arrival to the target floor.
		        \end{enumerate}}
			\item[•]{\textbf{Exceptions:} 
			    \begin{enumerate}
		        	\item Alarm is triggered \textit{before} entering elevator: NEG is denied access to all elevators.
		        	\item Alarm is triggered \textit{during} or \textit{after} entering elevator: The "open door" button is disabled.
		        	ECS enters alarm mode. Emergency key must be inserted to put the ECS in emergency mode, from which normal mode may
		        	be accessed.
		        \end{enumerate}}
			\item[•] \textbf{Priority:} Essential, must be implemented
			\item[•] \textbf{When available:} Always
			\item[•] \textbf{Frequency of use:} Many times per day
			\item[•] \textbf{Channel to actor:} Via bay button panel, bay doors, elevator doors, cabin button panel
			\item[•] \textbf{Secondary actors:} Mechanical technician, ECS technician, Emergency Personnel, another 
			generic human actor		
			\item[•]{\textbf{Channels to secondary actors:}
			    \begin{itemize}
					\item[] Mechanical Technician: cabin panel
					\item[] ECS Technician: cabin panel
					\item[] Emergency Personnel: alarm system, cabin panel, emergency key panel	    
			    \end{itemize}}
			\item[•]{\textbf{Open Issues:}
				\begin{enumerate}
					\item What happens if there is an obstruction while the ECS is in alarm mode?
				\end{enumerate}} 
		\end{itemize}
	
		\par\noindent\rule{\textwidth}{0.4pt}
		\begin{itemize} % go to another floor
			\item[] \textbf{Use Case:} \textit{GoToAnotherFloor}
			\item[] \textbf{Primary actor:} Executive Guest (EG)
			\item[] \textbf{Goal in context:} To travel from the current floor to the executive suite (top floor).
			\item[] \textbf{Preconditions:} System is not in maintenance, nor emergency mode, nor alarm mode. 
			Elevators, bays, and all mechanical parts can be in any safe state (aligned, below weight capacity, 
			no obstructions, etc.).
			\item[] \textbf{Trigger:} The EG decides to travel to his or her penthouse (the executive suite).
			\item[] { \textbf{Scenario:}
		        \begin{enumerate}
		        	\item EG observes the elevator bays.
		        	\item EG presses one or more buttons to request an elevator.
		        	\item EG enters first elevator to open its doors.
		        	\item EG inserts the executive key into the corresponding panel or scans the executive RFID card.
		        	\item EG presses one or more buttons at least one of which is the newly activated penthouse button (one or more 
		        	button presses may have been erroneous).
		        	\item{EG sees someone running toward the elevator
		        		\begin{enumerate}
		        			\item EG presses the "open doors" button
		        			\item EG presses the  "close doors" button
		        		\end{enumerate}}
		        	\item EG travels to target floor with zero or more stops in between.
		        	\item EG exits the elevator upon arrival to the target floor.
		        \end{enumerate}}
			\item[•]{\textbf{Exceptions:} 
			    \begin{enumerate}
		        	\item Alarm is triggered \textit{before} entering elevator: EG is denied access to all elevators.
		        	\item Alarm is triggered \textit{during} or \textit{after} entering elevator: The "open door" button is disabled.
		        	ECS enters alarm mode. Emergency key must be inserted to put the ECS in emergency mode, from which normal mode may
		        	be accessed.
		        	\item invalid executive key or executive RFID card: EG must contact hotel staff to resolve the issue.
		        \end{enumerate}}
			\item[•] \textbf{Priority:} Essential, must be implemented
			\item[•] \textbf{When available:} Always
			\item[•] \textbf{Frequency of use:} Many times per day, but fewer than the analogous NEG scenario
			\item[•] \textbf{Channel to actor:} Via bay button panel, bay doors, elevator doors, cabin button panel, key panel
			\item[•] \textbf{Secondary actors:} Mechanical technician, ECS technician, Emergency Personnel, another 
			generic human actor		
			\item[•]{\textbf{Channels to secondary actors:}
			    \begin{itemize}
					\item[] Mechanical Technician: cabin panel
					\item[] ECS Technician: cabin panel
					\item[] Emergency Personnel: alarm system, cabin panel, emergency key panel	
					\item[] Generic human actor: bay doors, cabin doors    
			    \end{itemize}}
			\item[•]{\textbf{Open Issues:}
				\begin{enumerate}
					\item How to deal with NEG that may "piggyback" on the EG's access?
				\end{enumerate}} 
		\end{itemize}	
		
		\par\noindent\rule{\textwidth}{0.4pt}
		\begin{itemize} % call front desk
			\item[•] \textbf{Use Case:} \textit{CallFrontDesk}
			\item[•] \textbf{Primary actor:} Non-Executive Guest (NEG)
			\item[•] \textbf{Goal in context:} To communicate with the front desk personnel.
			\item[•] \textbf{Preconditions:} Intercom system is functioning properly.
			\item[•] \textbf{Trigger:} The NEG has a question, comment or concern while inside an elevator cabin.
			\item[•]{\textbf{Scenario:}
		        \begin{enumerate}
		        	\item NEG observes "call button."
		        	\item NEG presses the button and waits for someone to answer.
		        	\item NEG has a conversation that addresses the question, comment, or concern.
		        	\item NEG ends the conversation.
		        	\item NEG continues traveling in the cabin or exits at some floor.
		        \end{enumerate}}
			\item[•]{\textbf{Exceptions:} 
			    \begin{enumerate}
		        	\item Front Desk Personnel does not answer: intercom call times out. NEG may try again.
		        	\item Alarm is triggered: alarm mode is entered, intercom remains functional.
		        \end{enumerate}}
			\item[•] \textbf{Priority:} Essential, must be implemented
			\item[•] \textbf{When available:} Always
			\item[•] \textbf{Frequency of use:} A moderate number of times per day
			\item[•] \textbf{Channel to actor:} Via bay button panel, intercom system
			\item[•] \textbf{Secondary actors:} Front desk personnel		
			\item[•]{\textbf{Channels to secondary actors:} 
			    \begin{itemize}
			   		 \item[] Front Desk Personnel: intercom system
                \end{itemize}}
			\item[•]{\textbf{Open Issues:}
				\begin{enumerate}
					\item How are calls from multiple cabins handled?
				\end{enumerate}}
		\end{itemize}
		
		\par\noindent\rule{\textwidth}{0.4pt}
		\begin{itemize} % clean cabin
			\item[•] \textbf{Use Case:} \textit{CleanCabin}
			\item[•] \textbf{Primary actor:} Custodial Personnel (CP)
			\item[•] \textbf{Goal in context:} To perform cabin cleanliness upkeep or to clean spills.
			\item[•] \textbf{Preconditions:} System is not in maintenance, nor emergency mode, nor alarm mode. 
			Elevators, bays, and all mechanical parts are in a safe state (aligned, below weight capacity, 
			no obstructions, etc.).
			\item[•] \textbf{Trigger:} CP is scheduled to perform upkeep or a spill occurs within a cabin.
			\item[•]{\textbf{Scenario:}
		        \begin{enumerate}
		        	\item CP is notified of the cabin in question.
		        	\item CP presses button to request the specific elevator.
		        	\item CP enters the cabin and waits for non-custodial personnel to exit the cabin.
		        	\item CP inserts custodial key to enter custodial mode.
		        	\item CP performs routine cleaning or spill.
		        	\item CP removes custodial key.
		        	\item CP exits a spotless cabin.
		        \end{enumerate}}
			\item[•]{\textbf{Exceptions:} 
			    \begin{enumerate}
		        	\item Maintenance key is invalid: custodial personnel must acquire a valid custodial key.
			    	\item Alarm is triggered: alarm mode is entered and custodial key functions are overridden.
		        \end{enumerate}}
			\item[•] \textbf{Priority:} Essential, must be implemented
			\item[•] \textbf{When available:} Always
			\item[•] \textbf{Frequency of use:} On regularly scheduled intervals and on random occasion
			\item[•] \textbf{Channel to actor:} Via bay button panel, bay doors, elevator doors, key panel, cabin button panel 
			\item[•] \textbf{Secondary actors:} Mechanical technician, ECS technician, Emergency Personnel, another 
			generic human actor.
			\item[•]{\textbf{Channels to secondary actors:}
			    \begin{itemize}
			    	\item[] Mechanical Technician: cabin panel
					\item[] ECS Technician: cabin panel
					\item[] Emergency Personnel: alarm system, cabin panel, emergency key panel
					\item[] Generic Human Actor: bay doors, cabin doors
			    \end{itemize}}
			\item[•]{\textbf{Open Issues:}
				\begin{enumerate}
					\item None known.
				\end{enumerate}}
		\end{itemize}

		\par\noindent\rule{\textwidth}{0.4pt}
		\begin{itemize} % move guest waste and laundry
			\item[•] \textbf{Use Case:} \textit{MoveGuestWasteAndLaundry}
			\item[•] \textbf{Primary actor:} Custodial Personnel (CP) 
			\item[•] \textbf{Goal in context:} To transfer guest laundry and guest waste while minimizing visibility.
			\item[•] \textbf{Preconditions:} System is not in maintenance, nor emergency mode, nor alarm mode. 
			Elevators, bays, and all mechanical parts are in a safe state (aligned, below weight capacity, 
			no obstructions, etc.).
			\item[•] \textbf{Trigger:} Custodial waste basket is full and/or custodial laundry basket is full.
			\item[•]{\textbf{Scenario:}
		        \begin{enumerate}
		        	\item CP runs out of space for guest waste and laundry in custodial cart.
		        	\item CP observes elevator bays.
		        	\item CP requests elevator and waits for one without guests.
		        	\item CP inserts custodial key to enter custodial mode.
		        	\item CP travels directly to the floor where the laundromat and waste disposal area are located.
		        	\item CP removes custodial key.
		        	\item CP exits cabin.
		        \end{enumerate}}
			\item[•]{\textbf{Exceptions:} 
			    \begin{enumerate}
		        	\item Maintenance key is invalid: custodial personnel must acquire a valid custodial key.
			    	\item Alarm is triggered: alarm mode is entered and custodial key functions are overridden.
		        \end{enumerate}}
			\item[•] \textbf{Priority:} Optional, may be implemented
			\item[•] \textbf{When available:} Always
			\item[•] \textbf{Frequency of use:} Varies with guest volume. The more guests, the more often this is used.
			\item[•] \textbf{Channel to actor:} Via bay button panel, bay doors, elevator doors, key panel, cabin button panel 
			\item[•] \textbf{Secondary actors:} Mechanical technician, ECS technician, Emergency Personnel, another 
			generic human actor
			\item[•]{\textbf{Channels to secondary actors:}
			    \begin{itemize}
			    	\item[] Mechanical Technician: cabin panel
					\item[] ECS Technician: cabin panel
					\item[] Emergency Personnel: alarm system, cabin panel, key panel
					\item[] Generic Human Actor: bay doors, cabin doors
			    \end{itemize}}
			\item[•]{\textbf{Open Issues:}
				\begin{enumerate}
					\item Should custodial personnel use separate elevators for this?
				\end{enumerate}}
		\end{itemize}
	
	
	\subsection*{Rare Scenarios} This section features scenarios that may occur occasionally. This includes mechanical and technical
	repair or maintenance and emergency situations. 
	
	\par\noindent\rule{\textwidth}{0.4pt}
		\begin{itemize} % go to another floor
			\item[] \textbf{Use Case:} \textit{RepairElevator}
			\item[] \textbf{Primary actor:} Mechanical Technician (MT)
			\item[] \textbf{Goal in context:} To fix mechanical problems along the shaft, within the bay, within the cabin, or among any
			motors.
			\item[] \textbf{Preconditions:} System is not in maintenance, nor emergency mode, nor alarm mode. 
			Elevators, bays. Mechanical state may or may not be in a safe state.
			\item[] \textbf{Trigger:} A mechanical failure occurs.
			\item[] { \textbf{Scenario:}
		        \begin{enumerate}
		        	\item MT is informed of the location of the problem.
		        	\item MT uses stairs to access entry point to the system.
		        	\item MT inserts maintenance key.
		        	\item MT fixes the issue.
		        	\item MT calibrates and resets sensors if necessary.
		        	\item MT tests the elevator by moving it through various floors.
		        	\item MT removes maintenance key and restores it to its normal functioning mode.
		        	\item MT exits the shaft, bay, or cabin where issue was found.
		        \end{enumerate}}
			\item[•]{\textbf{Exceptions:} 
			    \begin{enumerate}
		        	\item Alarm is triggered \textit{before} entering elevator: MT is denied access to all elevators.
		        	\item Alarm is triggered \textit{during} or \textit{after} entering elevator: The "open door" button is disabled.
		        	ECS enters alarm mode. Emergency key must be inserted to put the ECS in emergency mode, from which normal mode may
		        	be accessed. MT inserts maintenance key so emergency personnel will be aware of the situation.
		        	\item Maintenance key is invalid: Mechanical Technician must resolve the issue with hotel staff.
		        \end{enumerate}}
			\item[•] \textbf{Priority:} Essential, must be implemented
			\item[•] \textbf{When available:} Always
			\item[•] \textbf{Frequency of use:} A few times per month
			\item[•] \textbf{Channel to actor:} Via bay button panel, bay doors, elevator doors, cabin button panel, key panel
			\item[•] \textbf{Secondary actors:} Mechanical technician, Emergency Personnel, another 
			generic human actor		
			\item[•]{\textbf{Channels to secondary actors:}
			    \begin{itemize}
					\item[] Mechanical Technician: cabin panel
					\item[] Emergency Personnel: alarm system, cabin panel, emergency key panel
					\item[] Generic human actor: bay doors, cabin doors
			    \end{itemize}}
			\item[•]{\textbf{Open Issues:}
				\begin{enumerate}
					\item How to handle alarm mode that is triggered during mechanical maintenance?
				\end{enumerate}} 
		\end{itemize}
	
	\par\noindent\rule{\textwidth}{0.4pt}
		\begin{itemize} % go to another floor
			\item[] \textbf{Use Case:} \textit{RepairOrConfigureECS}
			\item[] \textbf{Primary actor:} ECS Technician (ET)
			\item[] \textbf{Goal in context:} To fix any issues related to the ECS or to configure it.
			\item[] \textbf{Preconditions:} System is not in maintenance, nor emergency mode, nor alarm mode. 
			Elevators, bays. Mechanical state is presumed to safe. ECS states may or may not be in a safe state.
			\item[] \textbf{Trigger:} An ECS failure occurs or configuration is sought.
			\item[] { \textbf{Scenario:}
		        \begin{enumerate}
		        	\item ET is informed of problem or prompted to configure the ECS.
		        	\item ET accesses the ECS which is centralized somewhere in the building.
		        	\item ET uses intercom to clear all cabins before disabling them and all bays.
		        	\item ET fixes the issue or configures the system.
		        	\item ET runs tests and performs any necessary calibration.
		        	\item ET starts the system and leaves the hotel.
		        \end{enumerate}}
			\item[•]{\textbf{Exceptions:} 
			    \begin{enumerate}
		        	\item Alarm is triggered \textit{before} disabling the system: ET must evacuate with everyone else.
		        	\item Alarm is triggered \textit{during} or \textit{after} disabling the system: ET must abort the 
		        	operation and evacuate.
		        	\item Access denied: ET must resolve the issue with the software company.
		        \end{enumerate}}
			\item[•] \textbf{Priority:} Essential, must be implemented
			\item[•] \textbf{When available:} Always
			\item[•] \textbf{Frequency of use:} A few times per month
			\item[•] \textbf{Channel to actor:} Via direct access to the ECS
			\item[•] \textbf{Secondary actors:} Generic human actor
			\item[•]{\textbf{Channels to secondary actors:}
			    \begin{itemize}
					\item[] Emergency Personnel: alarm system
					\item[] Generic human actor: intercom
			    \end{itemize}}
			\item[•]{\textbf{Open Issues:}
				\begin{enumerate}
					\item How to handle mechanical failure including ECS failure?
				\end{enumerate}} 
		\end{itemize}
	
	\par\noindent\rule{\textwidth}{0.4pt}
		\begin{itemize} % go to another floor
			\item[] \textbf{Use Case:} \textit{SearchAndRescue}
			\item[] \textbf{Primary actor:} Emergency Personnel (EP)
			\item[] \textbf{Goal in context:} Locate endangered human beings within the building as soon as possible.
			\item[] \textbf{Preconditions:} System is alarm mode. Mechanical and ECS states may or may not be safe.
			\item[] \textbf{Trigger:} The alarm system is triggered.
			\item[] { \textbf{Scenario:}
		        \begin{enumerate}
		        	\item EP receives alarm signal.
		        	\item EP rushes to to the building.
		        	\item EP accesses first available elevator, if any.
		        	\item EP inserts emergency key.
		        	\item EP controls the elevator to minimize search and rescue time.
		        	\item EP resolves endangerments after some indefinite repetition of the previous step.
		        	\item EP removes the emergency key and restores the system to its normal functioning state.
		        \end{enumerate}}
			\item[•]{\textbf{Exceptions:} 
			    \begin{enumerate}
		        	\item Access denied: EP must use the stairs.
		        \end{enumerate}}
			\item[•] \textbf{Priority:} Essential, must be implemented
			\item[•] \textbf{When available:} Always
			\item[•] \textbf{Frequency of use:} A few times per year
			\item[•] \textbf{Channel to actor:} Via cabin key panel
			\item[•] \textbf{Secondary actors:} Generic human actor
			\item[•]{\textbf{Channels to secondary actors:}
			    \begin{itemize}
					\item[] Generic human actor: bay doors, cabin doors
			    \end{itemize}}
			\item[•]{\textbf{Open Issues:}
				\begin{enumerate}
					\item How to handle mechanical failure or ECS failure during emergency operation?
				\end{enumerate}} 
		\end{itemize}

\section{Design Constraints} \label{cons}
\paragraph{} \textit{Due to the real-time nature of the system, there exist some additional constraints\footnote{Design Constraints by Anas.}. Namely, it must be
the case that all data structures concerning the safety controls are as fast as possible but also that they are capable of 
prioritizing all signals in the best way possible.}

	\subsection{Safety}
		\paragraph{} The safety is highly prioritized in our design of the ECS. We have 
		considered associating the fire alarm directly with the ECS because in the case 
		of the fire alarm triggering, this event has a very high priority and the ECS 
		should immediately react to this event. When it comes to elevator and bay safety, 
		each elevator and bay will have set of Safety Controls that they can communicate 
		through and the ECS monitors the situation from the top. By ensuring safety for 
		both bays and the elevators, the ECS operation will be carried smoothly without 
		hurting the passengers.

	\subsection{Implementation Guidelines}
		\paragraph{} According to the design, we suggest programmers to use some sort of 
		Concurrent safe Messaging Queue for the communication between sensing objects and 
		their associated parent objects. We also recommend using concurrent safe Priority 
		Queue for the ECS, so the ECS can react based on the certain given priority. The 
		priority should be considered because in the case of an emergency, the priority 
		for that event should be at the very top so that the ECS should immediately react 
		to it by closing its doors and going to first floor if its not there already.


\section{Definition of Terms} \label{defs}

\paragraph{} \textit{The following is a list of definitions contain the most commonly used 
technical terms within this document, whose meaning may not be immediately apparent to the 
lay reader. Most definitions are defined by the authors for use within the context of this 
document. Some may originate from vocabulary shared across the general references cited \nocite{*}. 
In the event that a definition was taken directly from a source, it is followed by a citation.
\footnote{This list is mostly a reduction of the term list found in the preceding Software 
Design Specification document.}}

\pagebreak	
	\begin{list}{}{}
		\item{\textbf{Action (Do:)} The step(s) taken by the ECS to get to the other state.}
		\item{\textbf{Bay} The location of elevator access on a given floor.}
		\item{\textbf{Cabin} The interior of the elevator; also used as a synonym for elevator.}
		\item{\textbf{Call} To summon an elevator to a given floor or, from a controller’s perspective, a summon to a given floor.}
		\item{\textbf{Capacity} Used in reference to weight capacity (see Load), or physical capacity(the amount of space in an elevator).}
		\item{\textbf{Elevator Control System/Controller (ECS)} A system for regulating the movement of elevators as well as controlling the states (see State).}
		\item{\textbf{Emergency Personnel} Personnel including firefighters, police, and paramedics.}
		\item{\textbf{Event} It includes all signals, inputs, decisions, interrupts, transitions, and actions to or from users or external devices.}
		\item{\textbf{Executive} In the context of passengers: Any passenger granted access to floors other inaccessible to the general public. In the context of floors: A floor accessible only by executive passengers.}
		\item{\textbf{Incoming Events} The event(s) which the interface is taking as an input.}
		\item{\textbf{Load} The physical weight being carried by an elevator.}
        \item{\textbf{Message Queue} The process of sending and receiving messages to communicate between different objects.}
		\item{\textbf{Obstruction} In the context of elevator operation, an obstruction is any entity that is present within the doorway of the elevator during closing of the doors.}
		\item{\textbf{Outgoing Events} The event(s) which the interface is outputting to ECS.}
		\item{\textbf{Passenger} Any person riding an elevator.}
        \item{\textbf{Priority Queue} The data is transferred with a priority number for the ECS to react appropriately.}
		\item{\textbf{Radio Frequency Identification (RFID)} Short range radio identification; typically used as a means of exchanging information (usually for validation) across short distances.}
		\item{\textbf{Safely} Without causing harm to any person or thing.}
		\item{\textbf{Shaft}  The physical enclosure within which the elevator travels, spanning the desired distance of travel.}
		\item{\textbf{Sensor}  A device for detecting or measuring physical quantities.}
	\end{list}

\pagebreak
\bibliography{../../ReferenceMaterial/BibTeX/references}
% run latex, then bibtex, then quickbuild (on the tex file)
\end{document}
