\documentclass[12pt]{article}
\usepackage[english]{babel}
\usepackage[numbers]{natbib}
\usepackage{graphicx}
\usepackage{xcolor}
\usepackage{sectsty}
\usepackage{float}
\bibliographystyle{apalike}
\setcitestyle{open={[},close={]}}
\sectionfont{\color{DarkBlue}} 
\subsectionfont{\color{LightBlue}}
\subsubsectionfont{\color{LightBlue}}
\paragraphfont{\color{LightBlue}}
\subparagraphfont{\color{LightBlue}}
\begin{document}
\definecolor{DarkBlue}{HTML}{4a5a8a}
\definecolor{LightBlue}{HTML}{4f81bf}
\begin{titlepage}
\begin{flushleft}
\vspace*{1cm}
\Huge
\textbf{Cretaceous Gardens Controller}\\
\vspace{1cm}
\Huge
\textit{Requirements Definition Document}\\
\vspace{1cm}
\Large
\textit{RDD Version 1.0}\\
\vspace{5cm}
\LARGE
Team \#3\\
17 October 2019
\vfill
\Huge
\textbf{CS 460 Software Engineering}
\end{flushleft}
\end{titlepage}
\normalsize
\tableofcontents
\newpage
\section{Introduction} %Zeke
\paragraph{} \textit{//introduction}

\section{Objectives} 
\label{obj}
\paragraph{} \textit{Four objectives believed to be critical for an 
optimal implementation of a \textit{Cretaceous Gardens Controller} are identified here 
\footnote{Objectives by Anas and Siri.}.}
 
	\subsection{Safety}\label{saf}
	\paragraph{}The main objective of the CGC is to provide a safe and reliable 
	experience for the client and the end users. Whether it be electric fences 
	or autonomous vehicles, ensuring safety is of highest priority. The end user
	ought to feel completely safe as should the client whose liability depends on
	this aspect.

	\subsection{User Experience}\label{use}
	\paragraph{} In order to fully realize an amazing experience for the end user,
	the CGC must facilitate token purchases and foster intuitive and seamless interactions
	with the vehicles. 

	\subsection{Maintainability}\label{mai}
	\paragraph{} For the sake of maintainability, the state the CGC should be easily
	accessible and it should be understandable. All nodes should inherit this feature. 
	The system should also be maintainable in real-time, so it should be prepared for
	any redundancies that support this aim.
	
	\subsection{Efficiency}\label{eff}
	\paragraph{}When it comes to efficiency, the CGC will make sure that both 
	the software and hardware components are highly efficient and functional. 
	Whether we talk about self-driving cars, pay kiosks, camera system, GPS, 
	or electric fences, the CGC must be efficient in interacting with them. 
	This will be possible when all the other objectives are met.  



\section{Overall System Organization} 
\label{sys}
\paragraph{} The CGC will be centralized \footnote{System Organization 
by Anas and Siri.} and will manage all relevant components. Figure 
\ref{fig:blackbox} shows a black box diagram of the CGC. The CGC receives inputs 
from sensors, user interfaces, and emergency systems like the \textit{Global 
Alarm System} and responds through appropriate output actions as described 
below. 
\begin{figure}[H]
	\centerline{\includegraphics[scale=.20]{CGCBlackBox.png}}
	\caption{A black box of high-level inputs and outputs of the \textit{CGC}.}
	\label{fig:blackbox}
\end{figure}
\vfill
\pagebreak

\section{Interfaces}
\label{int}
\paragraph{} \textit{The interfaces are broken\footnote{Interfaces by Siri 
and Anas.} up into main systems. They may be composed of their own 
sensors but said sensors do not interface with the CGC. The following list of interfaces 
list their sensors, hardware, and features.}

	\subsection{Pay Kiosk}
	\paragraph{} \textit{The purpose of the the Pay Kiosk interface is to connect 
	the physical Pay Kiosks to the CGC. It is composed of sensors and is designed 
	to do specific feature.}
			
	\paragraph{Sensors}
	\begin{list}{}{}
		\item \textbf{Touch Screen: }used to sense user interaction. 
		\item \textbf{Credit card: }accepts all major credit/debit cards. 
		\item \textbf{Cash receptacle: }accepts and analyzes cash. 
	\end{list}
		
	\paragraph{Hardware}
	\begin{list}{}{}
		\item \textbf{Change dispenser:} dispenses appropriate change to the 
		visitor buying a token.
		\item \textbf{Token dispenser: } dispenses token with unique ID to user.
	\end{list}

	\paragraph{Features}
	\begin{list}{}{}
		\item \textbf{Token builder:} Takes payment and the out user form and builds
		a unique token for the visitor.
		\item \textbf{Maintenance: } Enables employees to manage issues with kiosks 
		and provides machine health information. 
	\end{list}

	\subsection{Token}
	\paragraph{} \textit{The Token will act as an interface to multiple systems. 
	It will provide valuable information about the visitor and also interact with 
	the visitor. }
		
	\paragraph{Sensors}
	\begin{list}{}{}
		\item \textbf{Touch Screen: } interacts with the users. 
		\item \textbf{GPS: } senses the location of all tokens.
	\end{list}
		
	\paragraph{Hardware}
	\begin{list}{}{}
		\item \textbf{RFID: } the RFID chip will be programmed with a unique ID and 
		used for multiple purposes included access to various systems and areas.
		\item \textbf{Speaker: } the token contains speakers as hardware for alerts 
		and instructions.
	\end{list}

	\paragraph{Features}
	\begin{list}{}{}
		\item \textbf{Location/Map: }utilizes the GPS to provide location services.
	\end{list}

	\subsection{Car}
	\paragraph{} \textit{There will be an interface with all the cars. The autonomous 
	car will be built utilizing a partner. We will work closely with them to provide 
	access to specific sensors and features.}		
	
	\paragraph{Sensors}
	\begin{list}{}{}
		\item \textbf{RFID reader: }that covers the proximity of the car and is used 
		to grant access and count how many tokens are currently in the car. 
		\item \textbf{Seat Weight Sensor: }used to determine if there is someone 
		sitting in the seat. 
		\item \textbf{Camera: }used by the car for autonomous driving and also 
		connects to CGC for a needed scenario. 
		\item \textbf{Mic: }used to sense voice for use in an intercom.
	\end{list}
		
	\paragraph{Hardware}
	\begin{list}{}{}
		\item \textbf{Speaker: }used to alert guests.
		\item \textbf{Automatic Door Locks: }this will be initiated when the car 
		is determined to be moving.
		\item \textbf{Wireless networking: }for communication purposes to communicate 
		with the CGC.
	\end{list}
	
	\paragraph{Features}
	\begin{list}{}{}
		\item \textbf{Maintenance System: }allows for health checks and health status 
		communication of the car.  
	\end{list}

\section{Capabilities}
\label{cap}
\paragraph{}\textit{//section introduction}
% the outline for a solution
	\subsection{Capacity Protocol} \textit{//subsection introduction}
	\begin{enumerate}
		\item item
	\end{enumerate}

	\subsection{Emergency Protocol}
	\begin{enumerate}
		\item item
	\end{enumerate}

	\subsection{Efficient Usage Protocol}
	\begin{enumerate}
		\item item
	\end{enumerate}

	\subsection{Executive Usage Protocol}
	\begin{enumerate}
		\item item
	\end{enumerate}

	\subsection{Obstruction Protocol} 
	\begin{enumerate}
		\item item
	\end{enumerate}

\section{Design Constraints} %Matt
\label{con}
%restrictions placed on the solution space
\paragraph{} \textit{//section introduction}
	\subsection{General}
	\begin{itemize}
		\item item
	\end{itemize}
	
	\subsection{Safety}
	\begin{itemize}
		\item item
	\end{itemize}

\section{Definition of Terms} %Zeke
\label{def}
\textit{//section introduction}
% the basis for accurate communication
\bibliography{../../ReferenceMaterial/BibTeX/references}
% run latex, then bibtex, then quickbuild all on the tex file
\end{document}
